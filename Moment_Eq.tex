\section{Hierarchy of Moment Equations for Flows}
\begin{frame}{Ansatz for Derivation of Moment Equations}
	\scriptsize
    Consider approximation of the form
	\begin{align}
		\textcolor{cyan}{f(\textbf{x}, t, \phi, \theta) \approx f^N(\textbf{x},t,\phi,\theta) :=  \sum_{n=1}^{N} \sum_{i=-2n}^{2n} c^i_{2n}(\textbf{x},t) \cdot P^i_{2n}(\phi, \theta)}, \label{spectralmethod}
	\end{align}
	where $P^i_{2n}(\phi, \theta)$, $n = 0, \ldots, N$, $i = -2n, \ldots, 2n$
	\begin{itemize}
		\item are harmonic polynomial basis functions, i.e., the eigenfunctions of the Laplace-Beltrami operator with the eigenvalue $-2n(2n+1)$
		\pause
		\item form an orthonormal basis, wrt. the L2-inner product on the sphere
		\begin{align*}
			(g,h)_{S^2} := \int_{0}^{2\pi} \int_{0}^{\pi} g(\phi, \theta) h(\phi, \theta) \cdot \sin(\theta) d\theta d\phi,
		\end{align*}
	\pause
	\item and every square integrable function on $S^2$ can be expressed as a linear combination of spherical harmonics $f(\phi, \theta) = \sum^{\infty}_{n=1} \sum_{i=-2n}^{2n} c^i_{2n} \cdot P^i_{2n}$
	\end{itemize}
\end{frame}

\begin{frame}{Derivation of Moment Equations}
	\scriptsize
	We derive the moment equations by
	\begin{itemize}
		\item Insert ansatz for $f$ into kinetic equation from (\ref{SmochEq_wx}) 
\begin{align*}
\sin\theta \partial_{t}f^N(x,t,\phi,\theta) + &  \textcolor{red}{\partial_x (\cos\phi \cos\theta \sin^2\theta f^N)} \\
	& 
	 =- \textcolor{blue}{\partial_\theta \left(w_x \sin^3 \theta \cos \phi f^N\right)} +\textcolor{blue}{D_{r} \left( \partial_\phi \left(\frac{1}{\sin \theta} \partial_\phi f^N \right) + \partial_\theta (\sin \theta \partial_\theta f^N) \right)}
\end{align*}
		\item Multiply consecutively with all basis functions used in the ansatz and integrate the resulting equations over $\phi$ and $\theta$
	\end{itemize}
\pause
The system of the moment equations is expressed as follows
\begin{align}
	\partial_t Q(\boldsymbol{x},t) + \textcolor{red}{A}\partial_x Q(\boldsymbol{x},t) = \textcolor{blue}{\varphi} (Q(\boldsymbol{x},t)),
\end{align}
where $Q(x,t)=(c^0_0(x,t), c^{-2}_2(x,t), \ldots, c^{2N}_{2N}(x,t))^T$ represents the vector of the moments. 
\end{frame}

\begin{frame}{Derivation: A Closer Look}
\scriptsize
Consider 
\begin{align*}
	\textcolor{cyan}{\underbrace{\sin\theta \partial_{t}f^N(x,t,\phi,\theta)}_{[1]}} + &  \underbrace{\partial_x (\cos\phi \cos\theta \sin^2\theta f^N)}_{[2]} \\
	& 
	= -\partial_\theta \left(w_x \sin^3 \theta \cos \phi f^N\right) + D_{r} \left(\partial_\phi \left(\frac{1}{\sin \theta} \partial_\phi f^N \right) + \partial_\theta (\sin \theta \partial_\theta f^N) \right)
\end{align*}
\pause
For $k=0, \ldots, N$, $l=-2k, \ldots, 2k$ we obtain for term $[1]$
\begin{align*}
	&\int_{0}^{2\pi} \int_{0}^{\pi} \textcolor{cyan}{\sin\theta \partial_t \left(\sum_{n=0}^{N} \sum_{i=-2n}^{2n} c^i_{2n}(\textbf{x},t) \cdot P^i_{2n}(\phi, \theta)\right)}P^l_{2k}(\phi, \theta) \\
	&= \sum_{n=0}^{N} \sum_{i=-2n}^{2n} \partial_t c^i_{2n}(\textbf{x},t) \int_{0}^{2\pi} \int_{0}^{\pi} \sin\theta P^i_{2n}(\phi, \theta) P^l_{2k}(\phi, \theta) d\phi d\theta \\
	&=  \sum_{n=0}^{N} \sum_{i=-2n}^{2n} \partial_t c^i_{2n}(\textbf{x},t) (P^i_{2n}(\phi, \theta) P^l_{2k}(\phi, \theta))_{S^2} =  \sum_{n=0}^{N} \sum_{i=-2n}^{2n} \partial_t c^i_{2n}(\textbf{x},t) \cdot \delta_{n,k} \delta_{i,l} \\
	&= \partial_t c^l_{2k}(\textbf{x},t)
\end{align*}
\pause
This corresponds to this term
$\textcolor{cyan}{\partial_t Q(\boldsymbol{x},t)} + A\partial_x Q(\boldsymbol{x},t) = \varphi (Q(\boldsymbol{x},t))$
\end{frame}

\begin{frame}
	\scriptsize
Consider term [2]
\begin{align*}
	\underbrace{\sin\theta \partial_{t}f^N(x,t,\phi,\theta)}_{[1]} + &  \textcolor{cyan}{\underbrace{\partial_x (\cos\phi \cos\theta \sin^2\theta f^N)}_{[2]}} \\
	& 
	= -\partial_\theta \left(w_x \sin^3 \theta \cos \phi f^N\right) + D_{r} \left(\partial_\phi \left(\frac{1}{\sin \theta} \partial_\phi f^N \right) + \partial_\theta (\sin \theta \partial_\theta f^N) \right)
\end{align*}
\pause
This corresponds to this term
$\partial_t Q(\boldsymbol{x},t) + \textcolor{cyan}{A\partial_x Q(\boldsymbol{x},t)} = \varphi (Q(\boldsymbol{x},t))$ \\
\vspace{2mm}
For $N=1$ the matrix $A$ has the form
\begin{equation}
	\left[\begin{array}{c:c}
		A_{0,0} & A_{0,1} \\
		\hdashline A_{0,1}^T & A_{1,1}
	\end{array}\right]=\left[\begin{array}{c:ccccc}
		0 & 0 & \frac{1}{\sqrt{15}} & 0 & 0 & 0 \\
		\hdashline 0 & 0 & \frac{1}{7} & 0 & 0 & 0 \\
		\frac{1}{\sqrt{15}} & \frac{1}{7} & 0 & \frac{\sqrt{3}}{21} & 0 & 0 \\
		0 & 0 & \frac{\sqrt{3}}{21} & 0 & 0 & 0 \\
		0 & 0 & 0 & 0 & 0 & \frac{1}{7} \\
		0 & 0 & 0 & 0 & \frac{1}{7} & 0
	\end{array}\right] .
\end{equation}
\end{frame}

\begin{frame}
	\scriptsize
	For N = 2 the symmetric matrix A has the structure
	\begin{equation*}
		\left[\begin{array}{c:c:c}
		\parbox[c]{0.5cm}{\centering A_{0,0}} & \parbox[c]{1cm}{\centering A_{0,1}} & \\
			\hdashline
		\parbox[c]{0.5cm}{\centering A_{0,1}^T} & \parbox[c]{2cm}{\centering A_{1,1}} & \parbox[c]{2cm}{\centering A_{1,2}} \\
		\hdashline
		& \parbox[c]{0.5cm}{\centering A_{1,2}^T} & \parbox[c]{2cm}{\centering A_{2,2}}
		\end{array}\right] .
	\end{equation*}
\textcolor{cyan}{For any N the system of moment equations is hyperbolic.}
\end{frame}

\begin{frame}{Hierarchy of Moment Equations for Shear Flow}
\scriptsize
For $N=1$ we obtain 
\begin{equation}
	\begin{aligned}
		&\partial_t (Q(\boldsymbol{x},t)) + \textcolor{red}{
			\begin{pmatrix}
				0 & 0 & \frac{\sqrt{15}}{15} & 0 & 0 & 0 \\
				0 & 0 & \frac{1}{7} & 0 & 0 & 0 \\
				\frac{\sqrt{15}}{15} & \frac{1}{7} & 0 & \frac{\sqrt{3}}{21} & 0 & 0 \\
				0 & 0 & \frac{\sqrt{3}}{21} & 0 & 0 & 0 \\
				0 & 0 & 0 & 0 & 0 & \frac{1}{7} \\
				0 & 0 & 0 & 0 & \frac{1}{7} & 0
		\end{pmatrix}} \cdot \partial_x
		(Q(\boldsymbol{x},t)) \\
		&= \textcolor{blue}{
			\begin{pmatrix}
				0 & 0 & 0 & 0 & 0 & 0 \\
				0 & -6D_r & \frac{2}{7}w_x & 0 & 0 & 0 \\
				-\frac{\sqrt{15}}{5}w_x & -\frac{5}{7}w_x & -6D_r & \frac{3\sqrt{3}}{7}w_x & 0 & 0 \\
				0 & 0 & -\frac{4\sqrt{3}}{7}w_x & -6D_r & 0 & 0 \\
				0 & 0 & 0 & 0 & -6D_r & -\frac{5}{7} w_x \\
				0 & 0 & 0 & 0 & \frac{2}{7}w_x & -6D_r
			\end{pmatrix}
		} \cdot (Q(\boldsymbol{x},t)).
	\end{aligned}
\end{equation}
\end{frame}


\begin{frame}{Hierarchy of Moment Equations for Rectilinear Flow}
	\scriptsize
Consider
\begin{align*}
	\partial_{t}\left(\sin \theta f\right) &+ \textcolor{red}{\partial_x(\cos\phi \sin\theta \cos\theta f)} + \textcolor{red}{\partial_y(\sin \phi \sin \theta \cos \theta f )} \\
&= - \textcolor{blue}{\partial_\theta\left(( w_x \sin^3 \theta \cos \phi + w_y\sin \phi \sin^3 \theta) f\right)} + \textcolor{blue}{D_{r}(\partial_{\theta\theta}\partial_{\phi\phi})f}.
\end{align*}
	For $N=1$ the system is given as
	\begin{equation}
		\partial_t \left(\begin{array}{c}
			f_0 \\
			c_2^{-2} \\
			c_2^{-1} \\
			c_2^0 \\
			c_2^1 \\
			c_2^2
		\end{array}\right) + \textcolor{red}{A} \cdot \partial_x
		\left(\begin{array}{c}
			f_0 \\
			c_2^{-2} \\
			c_2^{-1} \\
			c_2^0 \\
			c_2^1 \\
			c_2^2
		\end{array}\right) + \textcolor{red}{B} \cdot \partial_z \left(\begin{array}{c}
			f_0 \\
			c_2^{-2} \\
			c_2^{-1} \\
			c_2^0 \\
			c_2^1 \\
			c_2^2
		\end{array}\right) =  \textcolor{blue}{\varphi} (Q(\boldsymbol{x},t)),
	\end{equation}
\end{frame}

\begin{frame}
	\scriptsize
The matrix $\textcolor{red}{A}$ and the Matrix $\textcolor{red}{B}$ has the form \\
\vspace{8pt}

\[
\begin{minipage}{0.45\linewidth}
	\[
	A = \begin{pmatrix}
		\vspace{8pt}
		0 & 0 & \frac{\sqrt{15}}{15} & 0 & 0 & 0 \\
		\vspace{8pt}
		0 & 0 & \frac{1}{7} & 0 & 0 & 0 \\
		\vspace{8pt}
		\frac{\sqrt{15}}{15} & \frac{1}{7} & 0 & \frac{\sqrt{3}}{21} & 0 &  0 \\
		\vspace{8pt}
		0 & 0 & \frac{\sqrt{3}}{21} & 0 & 0 & 0 \\
		\vspace{8pt}
		0 & 0 & 0 & 0 & 0 & \frac{1}{7} \\
		\vspace{8pt}
		0 & 0 & 0 & 0 & \frac{1}{7} & 0
	\end{pmatrix},
	\]
\end{minipage}
\begin{minipage}{0.45\linewidth}
	\[
	B = \begin{pmatrix}
		\vspace{8pt}
		0 & 0 & 0 & 0 & \frac{\sqrt{15}}{15} & 0 \\
		\vspace{8pt}
		0 & 0 & 0 & 0 & -\frac{1}{7} & 0 \\
		\vspace{8pt}
		0 & 0 & 0 & 0 & 0 &  \frac{1}{7} \\
		\vspace{8pt}
		0 & 0 & 0 & 0 & \frac{\sqrt{3}}{21} & 0 \\
		\vspace{8pt}
		\frac{\sqrt{15}}{15} & -\frac{1}{7} & 0 & \frac{\sqrt{3}}{21} & 0 & 0 \\
		\vspace{8pt}
		0 & 0 & \frac{1}{7} & 0 & 0 & 0
	\end{pmatrix}
	\]
\end{minipage}
\]
\end{frame}

