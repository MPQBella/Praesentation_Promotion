\section{Hierarchy of Moment Equations for Shear Flow}
\begin{frame}{Hierarchy of Moment Equations for Shear Flow 1D}
	\scriptsize
	Consider the simplified shear flow problem
	\begin{align*}
	\partial_{t}\left(\sin \theta f\right)+ \textcolor{blue}{\partial_\theta\left( w_x\sin ^3 \theta \cos \phi f\right)} - \textcolor{red}{\partial_x(-\cos\phi  \cos\theta \sin\theta^2 f)}
	= \textcolor{blue}{D_{r}(\partial_{\theta\theta}\partial_{\phi\phi})f}.
	\end{align*}
\pause
	The system of the moment equations is expressed as follows
\begin{align}
	\partial_t Q(\boldsymbol{x},t) + \textcolor{red}{A}\partial_x Q(\boldsymbol{x},t) = \textcolor{blue}{\varphi} (Q(\boldsymbol{x},t)),
\end{align}
where $Q(x,t)=(c^0_0, c^{-2}_2, \ldots, c^{2N}_{2N})$ represents the vector of the moments. \\
\vspace{1mm}
\pause
For $N=1$ we obtain 
\begin{equation}
	\partial_t \left(\begin{array}{c}
		f_0 \\
		c_2^{-2} \\
		c_2^{-1} \\
		c_2^0 \\
		c_2^1 \\
		c_2^2
	\end{array}\right) + \textcolor{red}{\begin{pmatrix}
			\vspace{8pt}
			0 & 0 & \frac{\sqrt{15}}{15} & 0 & 0 & 0 \\
			\vspace{8pt}
			0 & 0 & \frac{1}{7} & 0 & 0 & 0 \\
			\vspace{8pt}
			\frac{\sqrt{15}}{15} & \frac{1}{7} & 0 & \frac{\sqrt{3}}{21} & 0 &  0 \\
			\vspace{8pt}
			0 & 0 & \frac{\sqrt{3}}{21} & 0 & 0 & 0 \\
			\vspace{8pt}
			0 & 0 & 0 & 0 & 0 & \frac{1}{7}\\
			\vspace{8pt}
			0 & 0 & 0 & 0 & \frac{1}{7} & 0
	\end{pmatrix}} \cdot \partial_x
	\left(\begin{array}{c}
		f_0 \\
		c_2^{-2} \\
		c_2^{-1} \\
		c_2^0 \\
		c_2^1 \\
		c_2^2
	\end{array}\right)  =  \textcolor{blue}{\varphi} (Q(\boldsymbol{x},t)).
\end{equation}
\end{frame}

	\begin{comment}
	\begin{align*}
		\partial_{t} f_0 = &-\textcolor{red}{\frac{\sqrt{15}}{15}\partial_x c^{-1}_2}\\
		\partial_{t} c^{-2}_2 = &-\textcolor{red}{\frac{1}{7} \partial_x c^{-1}_2} -\textcolor{blue}{6D_rc^{-2}_2}+ \textcolor{blue}{\frac{2}{7}w_xc^{-1}_2} \\
		\partial_{t} c^{-1}_2 = &-\textcolor{red}{\frac{\sqrt{15}}{15}\partial_x f_0} - \textcolor{red}{\frac{1}{7} \partial_x c^{-2}_2} - \textcolor{red}{\frac{\sqrt{3}}{21}\partial_x c^0_2} \\ &-\textcolor{blue}{\frac{5}{7}w_x c^{-2}_2} -\textcolor{blue}{6D_r c^{-1}_2} +\textcolor{blue}{\frac{3\sqrt{3}}{7}w_x c^0_2} \\
		\partial_{t} c^0_2 = &-\textcolor{red}{\frac{\sqrt{3}}{21}\partial_x c^{-1}_2} -\textcolor{blue}{\frac{4\sqrt{3}}{7}w_x c^{-1}_2} -\textcolor{blue}{6D_r c^0_2} \\
		\partial_{t} c^1_2 = &-\textcolor{red}{\frac{1}{7}\partial_x c^2_2} -\textcolor{blue}{-6D_r c^{1}_2} -\textcolor{blue}{\frac{5}{7}w_x c^{2}_2}\\
		\partial_{t} c^2_2 = &-\textcolor{red}{\frac{1}{7}\partial_x c^1_2} +\textcolor{blue}{\frac{2}{7}w_x c^{1}_2} -\textcolor{blue}{6D_r c^{2}_2}
	\end{align*}
\end{comment}

\begin{comment}
\begin{frame}
	\scriptsize
For $N=1$ we obtain 
\begin{equation}
	\partial_t \left(\begin{array}{c}
		f_0 \\
		c_2^{-2} \\
		c_2^{-1} \\
		c_2^0 \\
		c_2^1 \\
		c_2^2
	\end{array}\right) + \textcolor{red}{\begin{pmatrix}
	\vspace{8pt}
	0 & 0 & \frac{\sqrt{15}}{15} & 0 & 0 & 0 \\
	\vspace{8pt}
	0 & 0 & \frac{1}{7} & 0 & 0 & 0 \\
	\vspace{8pt}
	\frac{\sqrt{15}}{15} & \frac{1}{7} & 0 & \frac{\sqrt{3}}{21} & 0 &  0 \\
	\vspace{8pt}
	0 & 0 & \frac{\sqrt{3}}{21} & 0 & 0 & 0 \\
	\vspace{8pt}
	0 & 0 & 0 & 0 & 0 & \frac{1}{7}\\
	\vspace{8pt}
	0 & 0 & 0 & 0 & \frac{1}{7} & 0
\end{pmatrix}} \cdot \partial_x
	\left(\begin{array}{c}
		f_0 \\
		c_2^{-2} \\
		c_2^{-1} \\
		c_2^0 \\
		c_2^1 \\
		c_2^2
	\end{array}\right)  =  \textcolor{blue}{\varphi} (Q(\boldsymbol{x},t)).
\end{equation}
\end{frame}
\end{comment}


\begin{frame}{Hierarchy of Moment Equations for Shear Flow 2D}
	\scriptsize
Consider
\begin{align*}
	&\partial_{t}\left(\sin \theta f\right)+ \textcolor{blue}{\partial_\theta\left(( w_x \sin^3 \theta \cos \phi + w_y\sin \phi \sin^3 \theta) f\right)}\\
&+\textcolor{red}{\partial_x(\cos\phi \sin\theta \cos\theta f)} + \textcolor{red}{\partial_y(\sin \phi \sin \theta \cos \theta f )}
= \textcolor{blue}{D_{r}(\partial_{\theta\theta}\partial_{\phi\phi})f}.
\end{align*}
	For $N=1$ the system is given as
	\begin{equation}
		\partial_t \left(\begin{array}{c}
			f_0 \\
			c_2^{-2} \\
			c_2^{-1} \\
			c_2^0 \\
			c_2^1 \\
			c_2^2
		\end{array}\right) + \textcolor{red}{A} \cdot \partial_x
		\left(\begin{array}{c}
			f_0 \\
			c_2^{-2} \\
			c_2^{-1} \\
			c_2^0 \\
			c_2^1 \\
			c_2^2
		\end{array}\right) + \textcolor{red}{B} \cdot \partial_z \left(\begin{array}{c}
			f_0 \\
			c_2^{-2} \\
			c_2^{-1} \\
			c_2^0 \\
			c_2^1 \\
			c_2^2
		\end{array}\right) =  \textcolor{blue}{\varphi} (Q(\boldsymbol{x},t)),
	\end{equation}
\end{frame}

\begin{frame}
	\scriptsize
The matrix $\textcolor{red}{A}$ and the Matrix $\textcolor{red}{B}$ has the form \\
\vspace{8pt}

\[
\begin{minipage}{0.45\linewidth}
	\[
	A = \begin{pmatrix}
		\vspace{8pt}
		u & 0 & \frac{\sqrt{15}}{15} & 0 & 0 & 0 \\
		\vspace{8pt}
		0 & u & \frac{1}{7} & 0 & 0 & 0 \\
		\vspace{8pt}
		\frac{\sqrt{15}}{15} & \frac{1}{7} & u & \frac{\sqrt{3}}{21} & 0 &  0 \\
		\vspace{8pt}
		0 & 0 & \frac{\sqrt{3}}{21} & u & 0 & 0 \\
		\vspace{8pt}
		0 & 0 & 0 & 0 & u & \frac{1}{7} \\
		\vspace{8pt}
		0 & 0 & 0 & 0 & \frac{1}{7} & u
	\end{pmatrix},
	\]
\end{minipage}
\begin{minipage}{0.45\linewidth}
	\[
	B = \begin{pmatrix}
		\vspace{8pt}
		v & 0 & 0 & 0 & \frac{\sqrt{15}}{15} & 0 \\
		\vspace{8pt}
		0 & v & 0 & 0 & -\frac{1}{7} & 0 \\
		\vspace{8pt}
		0 & 0 & v & 0 & 0 &  \frac{1}{7} \\
		\vspace{8pt}
		0 & 0 & 0 & v & \frac{\sqrt{3}}{21} & 0 \\
		\vspace{8pt}
		\frac{\sqrt{15}}{15} & -\frac{1}{7} & 0 & \frac{\sqrt{3}}{21} & v & 0 \\
		\vspace{8pt}
		0 & 0 & \frac{1}{7} & 0 & 0 & v
	\end{pmatrix}
	\]
\end{minipage}
\]
\end{frame}

