\section{Background Knowledge}
\begin{frame}{Finite Volume Method}
	\scriptsize
	Consider spatially one-dimensional linear systems of hyperbolic
	conservation laws, i.e.
	\begin{align}
		q_t(x,t) + f(q(x,t))_x = 0, \label{conservationLaw}
	\end{align}
	where $q: \mathbb{R} \times \mathbb{R}^+ \rightarrow \mathbb{R}^m$ represents a vector of $m$ conserved quantities and $f(q(x,t)) := Aq(x,t)$ is the flux function. $A \in \mathbb{R}^{m \times m}$ is hyperbolic. \\
	\vspace{\baselineskip}
	The numerical solution is computed as
    \[
    Q^n_i \approx \frac{1}{\Delta x} \int_{x_{i-\frac{1}{2}}}^{x_{i+\frac{1}{2}}} q(x,t_n) \, dx = \frac{1}{\Delta x} \int_{C_i} q(x,t_n) \, dx,
    \]
	where $C_i := (x_{i-\frac{1}{2}}, x_{i+\frac{1}{2}})$ are the grid cells of length $\Delta x := x_{i+\frac{1}{2}} - x_{i-\frac{1}{2}}$.
\end{frame}



\begin{frame}{Wave-Propagation}
\scriptsize
Wave-Propagation Algorithm (\cite{zbMATH00994008}) for the equation (\ref{conservationLaw}).

\begin{itemize}
	\item Based on solving Riemann problems at the grid cell interfaces
	\item Solution of Riemann problem as set of $m$ waves:
    \[Q^n_i - Q^n_{i-1} = \sum_{p=1}^{m} \lambda^p_{i-1/2} r^p = \sum_{p=1}^{m+1} W^p_{i-1/2}\]
    \pause
	\item Calculate numerical solution $Q^{n+1}_i$ at time $t^{n+1}$:
	\begin{itemize}
		\item all right-going waves at left edge $x_{i-\frac{1}{2}}$
		\[\mathcal{A}^{+} \Delta Q_{i-\frac{1}{2}}=\sum_{p=1}^m\left(\lambda^p\right)^{+} \mathcal{W}_{i-\frac{1}{2}}^{p}\]
		\item all left-going waves at right edge $x_{i+\frac{1}{2}}$
		\[\mathcal{A}^{-} \Delta Q_{i+\frac{1}{2}}=\sum_{p=1}^m\left(\lambda^p\right)^{-} \mathcal{W}_{i+\frac{1}{2}}^{p}	\]
	\end{itemize}
   \pause
   \item \textbf{Update formula} for the cell average:
   \[Q^{n+1}_i = Q^n_i - \frac{\Delta t}{\Delta x} \left( A^+ \Delta Q_{i-1/2} + A^- \Delta Q_{i+1/2} \right)\]
\end{itemize}
\end{frame}

\begin{frame}{Solving Source Terms using Operator Splitting}
	\scriptsize
	Consider
	\begin{align}
		q_t + f(q)_x = \varphi(q) \label{conserLawSource}
	\end{align}
	
	Solve this problem (\ref{conserLawSource}) by splitting it into two subproblems of the form:
  \begin{enumerate}
  	\item $q_t + f(q)_x = 0$
  	\item $q_t = \varphi(q)$
  \end{enumerate}
  \vspace{2mm}
   Use finite volume methods for subproblem 1 and an ODE solver for subproblem 2.\\
  \vspace{4mm}
  \pause
  The \textbf{Strang Splitting} (\cite{zbMATH00994008}) approach involves:
  \begin{enumerate}
  	\item Solving subproblem 1 over a half time step of length $\Delta t/2$
  	\item Using the result as initial data for subproblem 2 over a full time step $\Delta t$
  	\item Finally, solve subproblem 1 again with another half time step of length $\Delta t/2$
  \end{enumerate}
\end{frame}