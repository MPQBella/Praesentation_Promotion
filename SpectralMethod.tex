\section{Spectral method}
\begin{frame}{Mathematical Model for the Sedimentation of Rod-Like Particles \cite{zbMATH06724175}}
	\scriptsize
	\text{Coupling of a Smoluchowski Equation and a Navier-Stokes Equation}
	\begin{align*}
		\partial_t f+\nabla_{\boldsymbol{x}} \cdot(\boldsymbol{u} f) & +\nabla_{\boldsymbol{n}} \cdot\left(P_{\boldsymbol{n}^{\perp}} \nabla_{\boldsymbol{x}} \boldsymbol{u} \boldsymbol{n} f\right)-\nabla_{\boldsymbol{x}} \cdot\left((I+\boldsymbol{n} \otimes \boldsymbol{n}) e_3 f\right) \\
		& =D_r \Delta_n f+\gamma \nabla_{\boldsymbol{x}} \cdot(I+\boldsymbol{n} \otimes \boldsymbol{n}) \nabla_{\boldsymbol{x}} f, \notag \\
		\sigma & =\int_{S^{d-1}}(\text{d} \; \boldsymbol{n} \otimes \boldsymbol{n}-I) f d \boldsymbol{n},  \\
		\operatorname{Re}\left(\partial_t \boldsymbol{u}+\left(\boldsymbol{u} \cdot \nabla_{\boldsymbol{x}}\right) \boldsymbol{u}\right) & =\Delta_{\boldsymbol{x}} \boldsymbol{u}-\nabla_{\boldsymbol{x}} p+\delta \gamma \nabla_{\boldsymbol{x}} \cdot \sigma-\delta \int_{S^{d-1}} f d \boldsymbol{n} \, e_3, \\
		\nabla_{\boldsymbol{x}} \cdot \boldsymbol{u} & =0,
	\end{align*}
	where $f = f(t; x; n), x \in \mathbb{R}^d , n \in  S^{d-1}, t \in \mathbb{R}$ is a density distribution function of particle orientation. $D_r, \gamma, \delta$ and $Re$ are non-dimensional values.
\end{frame}


\begin{frame}
	\scriptsize
	Consider 
	\begin{align}
		\textcolor{blue}{\partial_t f}+ \nabla_{\boldsymbol{x}} \cdot(\boldsymbol{u} f) & +\textcolor{blue}{\nabla_{\boldsymbol{n}} \cdot\left(P_{\boldsymbol{n}^{\perp}} \nabla_{\boldsymbol{x}} \boldsymbol{u} \boldsymbol{n} f\right)}-\nabla_{\boldsymbol{x}} \cdot\left((I+\boldsymbol{n} \otimes \boldsymbol{n}) e_3 f\right) \notag \\
		& = \textcolor{blue}{D_r \Delta_n f}+\gamma \nabla_{\boldsymbol{x}} \cdot(I+\boldsymbol{n} \otimes \boldsymbol{n}) \nabla_{\boldsymbol{x}} f. \label{SmochEq_kurz}
	\end{align}
	Rewrite the equation (\ref{SmochEq_kurz}) in spherical coordinates (\cite{zbMATH05037679}, \cite{zbMATH07295185}) and it follows
	\begin{align}
		\sin \theta \partial_t f + \partial_\phi\left(a(\phi, \theta) f\right)+\partial_\theta\left(b(\phi, \theta) f\right) = D_r \left(\partial_\phi\left(\frac{1}{\sin \theta} \partial_\phi f\right)+\partial_\theta\left(\sin \theta \partial_\theta f\right)\right), \label{Smochluch_S2}
	\end{align}
	with $\phi \in [0, 2 \cdot \pi]$ and $\theta \in [0, \pi]$.
	Solving the Smoluchoswki equation (\ref{Smochluch_S2}) on $S^2$ by using a \textcolor{cyan}{spectral method}, which is based on the ansatz
	\begin{align}
		f(\phi, \theta, t) = f_0(t) \cdot P_0^0 + \sum_{n=1}^{\infty} \sum_{i=-n}^{n} c^i_{2n}(t) \cdot P^i_{2n}(\phi, \theta), \label{ansatz}
	\end{align}
	where $P^i_{2n}(\phi, \theta)$ are harmonic polynomial basis functions. %, i.e. are eigenfunctions of Laplace Beltrami operator.
\end{frame}

%%%%%%%%%%%%%%%%%%%%%%
% Properties
%%%%%%%%%%%%%%%%%%%%%%

\begin{frame}{Harmonic polynomial basis functions}
	\scriptsize
	Let ${P}^{i}_{n}(\phi, \theta)$ with $n = 0, ..., \infty$ and $i = n, ..., -n$ be the basis function
	\begin{align*}
		TODO
	\end{align*}
	The scalar product of any two basis functions over sphere is defined as follows
	\begin{align*}
		<P^i_n, P^l_m>_{S^2} = \int_{0}^{2\pi} \int_{0}^{\pi} P^{i}_{n}(\phi, \theta) \cdot P^{l}_{m}(\phi, \theta) \cdot \sin(\theta) d\theta d\phi.
	\end{align*}
\end{frame}

\begin{frame}{Properties of harmonic polynomial basis functions}
	\scriptsize
	\begin{block}{Property 1}
		Let $P^{i}_{n}(\phi, \theta)$ and $P^{l}_{m}(\phi, \theta)$ are two different harmonic polynomial basis functions. Then
		\begin{align*}
			<P^i_n, P^l_m>_{S^2} = 0,
		\end{align*}
	for $i \neq l$ or $n \neq m$.
	\end{block}
	
	\begin{block}{Property 2}
		Let $P^i_{n}$ be the normalized harmonic polynomial basis functions. Then
		\begin{align*}
			<P^i_n, P^i_n>_{S^2} = 1.
		\end{align*}
	\end{block}

	\begin{block}{Property 3}
		The spherical harmonic function are the eigenfunctions of Laplace-Beltrami operator with the eigenvalues $(- n (n +1),n \in \mathbb{N}_0)$(see \cite{zbMATH01218597})
		\begin{align*}
			\Delta_{S^2}P^i_n = -n(n+1) P^i_n.
		\end{align*}
	\end{block}
\end{frame}


%%%%%%%%%%%%%%%%%%%%%%%%%%%%%%%%%%%%%%
% Back to the spectral method
%%%%%%%%%%%%%%%%%%%%%%%%%%%%%%%%%%%%%%

\begin{frame}{Spectral method}
	\centering
	\scriptsize
	Recall the Smochluchowski equation on $S^2$
	\begin{align}
		\underbrace{\sin \theta \partial_t f}_{(1)} + \underbrace{\partial_\phi\left(a(\phi, \theta) f\right)+\partial_\theta\left(b(\phi, \theta) f\right)}_{(2)} = \underbrace{D_r \left(\partial_\phi\left(\frac{1}{\sin \theta} \partial_\phi f\right)+\partial_\theta\left(\sin \theta \partial_\theta f\right)\right)}_{(3)} \label{Smoch_S2}
	\end{align}	
	and the ansatz
	\begin{align}
		f(\phi, \theta, t) = f_0(t) \cdot P_0^0 + \sum_{n=1}^{\infty} \sum_{i=-n}^{n} c^i_{2n}(t) \cdot P^i_{2n}(\phi, \theta).
	\end{align}
\end{frame}
\begin{frame}
	\scriptsize
	
	The Laplace Beltrami operator (\cite{zbMATH07295185}) on the unit sphere $S^2$ is given by
	\begin{align}
		\Delta_{S^2} f = \frac{1}{\sin ^2 \theta} \partial_{\phi \phi} f + \frac{1}{\sin \theta} \partial_\theta\left(\sin \theta \partial_\theta f\right). \label{laplace_eq}
	\end{align}
	From property 3 and the equation $(\ref{laplace_eq})$, it follows for the term (3)
	\begin{align*}
		\Delta_{S^2} P^i_{2n} = -n(n+1)P^i_{2n}.
	\end{align*}
	For the term $(1)$ and $(2)$, we inserting (5) in (4), multiplying with each basis function and integrate it over $S^2$. We derive a system of ODEs for the coefficients
	\scriptsize
	\begin{equation}
		\left(\begin{array}{c}
			f_0 \\
			c_2^{-2} \\
			\vdots \\
			c_{2n}^i
		\end{array}\right)^{\prime}=A\left(\begin{array}{c}
			f_0 \\
			c_2^{-2} \\
			\vdots \\
			c_{2n}^i
		\end{array}\right),
	\end{equation}
with $A \in \mathbb{R}^{cnxcn}$
\begin{equation*}
	c n= \begin{cases}\text { order }=2: & cn= 2 \cdot \text {order}+2 \\ \text { order }=\text {even: } & c2=6 \\  &c n=c2 \\ &c n=c2+\sum_{i=4}^{order}(2 i+1)\end{cases}
\end{equation*}
\end{frame}

%%%%%%%%%%%%%%%%%%
% Example
%%%%%%%%%%%%%%%%%%
\begin{frame}
	\centering
	Example: Shear flow
\end{frame}

\begin{frame}{Example: Shear flow}
	\scriptsize
	Consider the Smoluchowski equation $(\ref{Smoch_S2})$ with the velocity gradient 
	\begin{align*}
		\vec{u}=\left(\begin{array}{l}
			u(x,y,z) \\
			v(x,y,z) \\
			w(x,y,z)
		\end{array}\right),
		\nabla_x \vec{u}_{\mathrm{ext}}=\left(\begin{array}{lll}
			u_{x} & u_{y} & u_{z} \\
			v_{x} & v_{y} & v_{z} \\
			w_{x} & w_{y} & w_{z}
		\end{array}\right)=\left(\begin{array}{ccc}
			0 & 1 & 0 \\
			0 & 0 & 0 \\
			0 & 0 & 0
		\end{array}\right) .
	\end{align*}
	With the given velocity gradient it follows
	\begin{align}
		\partial_{t}\left(\sin \theta f\right) &+ \partial_\theta\left(\sin \phi \cos \phi \sin ^2 \theta \cos \theta f\right)+ \partial_\phi\left(- \sin \theta \sin ^2 \phi f \right) \nonumber \\
		&=D_{r}\left(\partial_\theta \left(\sin \theta \partial_\theta f\right)+ \partial_\phi\left(\frac{1}{\sin \theta} \partial_\phi f\right)\right). \label{smoEq} 
	\end{align}
	Consider the ansatz with the zeroth order
	\begin{align}
		f(\phi, \theta, t)= f_0(t) \cdot P_0^0 \label{ansatz_0nd}.
	\end{align}

	Insert the ansatz (\ref{ansatz_0nd}) in (\ref{smoEq})
	\begin{align}
		\partial_{t}(f_0(t) \cdot P_0^0)+\frac{1}{\sin \theta}\left(\partial_\theta(\ldots)+\partial_\phi(\ldots)\right)=\frac{1}{\sin \theta} D_r \left(\ldots \right) \label{eq_mitAnsatz}.
	\end{align}
\end{frame}

\begin{frame}
	\scriptsize
	We know
	\begin{align*}
		\Delta_{S^2} P_0^0(\phi, \theta) = \frac{1}{\sin \theta} D_r (\ldots) = \lambda_{2n,i} \cdot P^{0}_0(\phi, \theta),
	\end{align*}
	where $\lambda_{2n,i}$ is the corresponding eigenvalue.\\
	Since $P_0^0(\phi, \theta) = 1$ does not depend on $\phi$ and $\theta$, the partial derivatives will be zero
	\begin{align}
		\Delta_{S^2} P_0^0(\phi, \theta) = 0.
	\end{align}
	Consider the rest of the equation (\ref{eq_mitAnsatz})
	\begin{align*}
		\underbrace{\partial_{t}(f_0(t) \cdot P_0^0)}_{(1)}+ \underbrace{\frac{1}{\sin\theta}\left( \partial_\theta(\sin \phi \cos \phi \sin ^2 \theta \cos \theta \cdot f_0(t) \cdot P_0^0)+ \partial_\phi(- \sin \theta \sin ^2 \phi \cdot f_0(t) \cdot P_0^0)\right)}_{(2)}.
	\end{align*}
	It is
	\begin{align*}
		\partial_t\left(f_0(t) \cdot P^0_0\right)=f_0^{\prime}(t).
	\end{align*}
	Let $z(\phi, \theta) := (2)$
\end{frame}

\begin{frame}
	\scriptsize
	Project the solution $z(\phi, \theta)$ onto all polynomials to find out which polynomial are needed
	\begin{align*}
		\int_{0}^{2\pi} \int_{0}^{\pi} z(\phi, \theta) \cdot P^{-2}_2(\phi, \theta) \, \sin \theta d\theta d\phi &\overset{Maple}{=} 0 \\
		\int_{0}^{2\pi} \int_{0}^{\pi} z(\phi, \theta) \cdot P^{-1}_2(\phi, \theta) \, \sin \theta d\theta d\phi &\overset{Maple}{=} 0 \\
		\int_{0}^{2\pi} \int_{0}^{\pi} z(\phi, \theta) \cdot P^{0}_2(\phi, \theta) \, \sin \theta d\theta d\phi &\overset{Maple}{=} 0 \\
		\int_{0}^{2\pi} \int_{0}^{\pi} z(\phi, \theta) \cdot P^{1}_2(\phi, \theta) \, \sin \theta d\theta d\phi &\overset{Maple}{=} 0 \\
		\int_{0}^{2\pi} \int_{0}^{\pi} z(\phi, \theta) \cdot P^{2}_2(\phi, \theta) \, \sin \theta d\theta d\phi &\overset{Maple}{=} -\frac{\sqrt{15}}{5}
	\end{align*}
\end{frame}

\begin{frame}
	\scriptsize
	It follows
	\begin{align}
		f_0(t) \cdot P^0_0 \cdot \frac{1}{\sin \theta}\left(\partial_\theta\left(\sin \phi \cos \phi \sin ^2 \theta \cos \theta\right)+\partial_\phi \left(-\sin \theta \sin ^2 \phi\right)\right) = f_0(t) \left[ -\frac{\sqrt{15}}{2} P^2_2 \right]. \label{teil1}
	\end{align}
	Together we have
	\begin{align*}
		f_0^{\prime}(t)-\frac{\sqrt{15}}{2} f_0(t) P_2^2(\phi,\theta) = 0\cdot  P^0_0(\phi, \theta)  D_r.
	\end{align*}
\end{frame}

\begin{frame}
	\scriptsize
	\centering
	For the ansatzfunction with higher order, the calculation is done in the same way. \\
	As an example we obtain an ODE system with ansatzfunction of the $2nd.$ order
	\begin{equation}
		\left(\begin{array}{c}
			f_0^{\prime}(t) \\
			c_2^{-2} \\
			c_2^{-1} \\
			c_2^0 \\
			c_2^1 \\
			c_2^2
		\end{array}\right)=\left(\begin{array}{cccccc}
			0 & 0 & 0 & 0 & 0 & 0 \\
			0 & -6 D_r & 0 & 0 & 0 & 1 \\
			0 & 0 & -6 D_r & 0 & 5/7 & 0 \\
			0 & 0 & 0 & -6 D_r & 0 & -\frac{\sqrt{3}}{7} \\
			0 & 0 & -2/7 & 0 & -6 D_r & 0 \\
			\frac{\sqrt{15}}{5} & 1 & 0 & -\frac{\sqrt{3}}{7} & 0 & -6 D_r
		\end{array}\right) \cdot\left(\begin{array}{c}
			f_0 \\
			c_2^{-2} \\
			c_2^{-1} \\
			c_2^0 \\
			c_2^1 \\
			c_2^2
		\end{array}\right)
	\end{equation}
\end{frame}

\begin{comment}
\begin{frame}
	\begin{table}[H]
		\scriptsize
		\begin{tabular}{|c|c|}
			\hline
			0th order	& 2nd order \\
			\hline
			& $P^{-2}_2 = \sqrt{\frac{15}{16\pi}}\sin^2(\theta)\cos(2\phi)$ \\
			& $P^{-1}_2 = \sqrt{\frac{15}{4\pi}}\sin(\theta)\cos(\theta)\cos(\phi)$ \\
			$P^0_0 = \sqrt{\frac{1}{4\pi}} \cdot 1$	& $P^0_2 = \sqrt{\frac{45}{16\pi}}\cos^2(\theta) - \frac{1}{3}$  \\
			&  $P^1_2 = \sqrt{\frac{15}{4\pi}}\sin(\theta)\cos(\theta)\sin(\phi)$\\
			&  $P^2_2 = \sqrt{\frac{15}{16\pi}}\sin^2(\theta)\sin(2\phi)$\\
			\hline
		\end{tabular}
		\caption{Normalized harmonic polynomial basis functions.}
	\end{table}
\end{frame}
\end{comment}

%\begin{frame}
%	\begin{itemize}
%		\item The harmonic polynomial basis function of the second order are the eigenfunctions of the Laplace Beltrami operator with eigenvalue $-6$. 
%		\item $P^{-4}_4, ..., P^4_4$ are eigenfunctions of the Laplace Beltrami operator with eigenvalue $-20$.
		%\item For higher order there exist also an eigenvalue relation the Laplace Beltrami operator.
%	\end{itemize}
%\end{frame}


