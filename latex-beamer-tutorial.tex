\documentclass[lightblue, notheorems, xcolor=dvipsnames]{beamer}
%\usepackage{beamerthemeshadow}
\usetheme{Madrid}
\setbeamercolor{author in head/foot}{bg=structure.fg, fg=white} % Hintergrundfarbe wie structure.fg und Textfarbe weiß

\setbeamertemplate{footline}{%
	\hbox{%
		\begin{beamercolorbox}[wd=\paperwidth,ht=2.25ex,dp=1ex,center]{author in head/foot}%
			\insertframenumber/\inserttotalframenumber
		\end{beamercolorbox}%
	}%
}

\usepackage[english]{babel}
\usepackage[utf8]{inputenc}
\usepackage[T1]{fontenc}
\usepackage{amsmath, amssymb, amsthm, amsfonts} %einige hilfreiche Mathe-Pakete
\setbeamertemplate{theorems}[ams style]
\usepackage{tabularx, multirow} %fuer Tabellen
\usepackage{graphicx, tikz} %Grafiken
\usepackage{setspace} %Paket f\"ur das Formatieren von Quellen unten auf einer Folie
\newcommand{\diff}{\mathop{}\!\mathrm{d}}
\usepackage[T1]{fontenc}

% Algorithm
\usepackage[ruled,vlined]{algorithm2e}
\usepackage{algpseudocode}
\usepackage{algorithmicx}



% BIlder
\setbeamertemplate{caption}[numbered]
\usepackage{subcaption}
%\usepackage{subfig}
\usepackage{graphicx}
\usepackage{float}
%\usepackage{tikz}

\usepackage{comment}

% Ausblenden Navigation
\beamertemplatenavigationsymbolsempty

% definiere Enviroments
\theoremstyle{definition}
\newtheorem{defi}{Definition}[section]
\newtheorem{idea}{Idee}[section]
\theoremstyle{plain}
\newtheorem{theo}[defi]{Satz}
\newtheorem{coroll}[defi]{Korollar}%
\newtheorem{lem}[defi]{Lemma}%
\newtheorem{aim}[defi]{Ziel}
%\newtheorem*{proof}{Beweis}%[theorem]

\theoremstyle{example}
\newtheorem{remark}[defi]{Bemerkung}%
\newtheorem{examp}[defi]{Beispiel}%
\newtheorem*{lsg}{Lösung}


\DeclareMathOperator*{\argmax}{argmax}


\iffalse
\expandafter\def\expandafter\insertshorttitle\expandafter{%
    \insertshorttitle\hfill%
    \insertframenumber\,/\,\inserttotalframenumber}
\fi



\SetKwRepeat{Do}{do}{while}%
\begin{document}
   %\maketitle  %erstellt Titelseite mit oben stehenden Informationen
  % optional: mit 
  \author{Bella My Phuong Quynh Duong}
  \date{September 23, 2024} %Datum des Vortrags eintragen
  \institute{Heinrich-Heine University D\"usseldorf}
  \title{Numerical Simulations of a Coupled Moment System for Modeling Sedimentation in Suspensions of Rod-Like Particles}
  
  \begin{frame}
  	\titlepage
  	\begin{figure}[htpb]
  		\begin{center}
  			\includegraphics[width=0.2\textwidth]{logo.png}
  		\end{center}
  	\end{figure}
  \end{frame}
\begin{frame}
   \frametitle{Table of Contents}
\tableofcontents %dieser Befehl erzeugt das Inhaltsverzeichnis
\end{frame}

\AtBeginSection[]
{
	\begin{frame}<beamer>
		\frametitle{Overview}
		\tableofcontents[currentsection]
	\end{frame}
}

\section{Introduction}

\begin{frame}{Mathematical Model for the Sedimentation of Rod-Like Particles}
	\scriptsize
\text{Coupling of a kinetic Smoluchowski equation with Navier-Stokes equation}
\begin{align*}
\partial_t f+ \nabla_{\boldsymbol{x}} \cdot(\boldsymbol{u} f) & +\nabla_{\boldsymbol{n}} \cdot\left(P_{\boldsymbol{n}^{\perp}} \nabla_{\boldsymbol{x}} \boldsymbol{u} \boldsymbol{n} f\right)- \nabla_{\boldsymbol{x}} \cdot\left((I+\boldsymbol{n} \otimes \boldsymbol{n}) \boldsymbol{e_3} f\right) \\
	& =D_r \Delta_n f, \\
	\operatorname{Re}\left(\partial_t \boldsymbol{u}+\left(\boldsymbol{u} \cdot \nabla_{\boldsymbol{x}}\right) \boldsymbol{u}\right) & =\Delta_{\boldsymbol{x}} \boldsymbol{u}-\nabla_{\boldsymbol{x}} p-\delta \left(\int_{S^{d-1}} f d \boldsymbol{n} \right) \boldsymbol{e_3}, \\
	\nabla_{\boldsymbol{x}} \cdot \boldsymbol{u} & =0,
\end{align*}
where $f = f(\boldsymbol{x}, t, \boldsymbol{n})$ represents the particle distribution of rod-like particles as a function of time $t$, space $\boldsymbol{x} \in \mathbb{R}^3$ and orientation $\boldsymbol{n} \in  S^2$. 
$D_r, \delta$ and $Re$ are non-dimensional parameters.
\begin{beamercolorbox}[sep=1em,wd=\linewidth,right]{}
	\tiny{Helzel $\&$ Tzavaras, 2017}
\end{beamercolorbox}
\end{frame}

\begin{frame}{Outline of the Project}
	\scriptsize
	\begin{block}{Goal}
		\begin{itemize}
			\item Reduce the high-dimensional kinetic equation (in space and orientation) to a lower-dimensional system of moment equations (in space).
			\item Derive and approximate hierarchies of moment equations for the coupled kinetic-fluid model with \textcolor{cyan}{$f$ on $S^2$}
		\end{itemize}
	\end{block}
    \pause
	\begin{block}{Our Approach}
		 Investigate different coupled flow situations
		\begin{itemize}
			\item externally imposed velocity field
            \item coupled problems: 1D shear flow, 2D rectilinear flow and
             \pause now 3D flow with periodic boundary conditions
		\end{itemize}
    \end{block}
\end{frame}

%----------------------------------------------------------
%----------------------------------------------------------


\begin{comment}
\begin{frame}
	\scriptsize
	\begin{figure}[H]
		\centering
		\begin{minipage}{0.4\textwidth}
			\includegraphics[scale=0.3]{Bilder_wxwy/t=0_wxwy=1_wxwy=-1}
		\end{minipage}
		\hfill 
		\begin{minipage}{0.4\textwidth}
			\includegraphics[scale=0.3]{Bilder_wxwy/t=200_wxwy=1_wxwy=-1}
		\end{minipage}
		\caption{Numerical results for $c^0_0$ at different times using $D_r=1$ and $w_x=w_y=1$ for $x<50$ and $w_x=w_y=-1$ otherwise. A cluster with higher particle density splits into two, each moving in opposite directions.}
		
	\begin{minipage}{0.4\textwidth}
		\centering
		\includegraphics[scale=0.25]{Bilder_wxwy/14th_t=100_mx=my=200_random_Dr=1_(1.d0+(1.d-2rand(0)-5.d-4))Divide(2.d0dsqrt(pi))}
		\end{minipage}
		\hfill 
		\begin{minipage}{0.4\textwidth}
			\centering
			\includegraphics[scale=0.25]{Bilder_wxwy/14th_t=200_mx=my=200_random_Dr=1_(1.d0+(1.d-2rand(0)-5.d-4))Divide(2.d0dsqrt(pi))}
		\end{minipage}
		\caption{Solution structure of \(c^0_0\) with \(N = 7\).}
	\end{figure}
\end{frame}
\end{comment}










\section{Shear Flow}
\begin{frame}{Shear Flow}
	\scriptsize
	For a \textit{shear flow} assume $	\boldsymbol{u} = (0,0,w(x,t))^T$, $f=f(x,t,\phi,\theta)$. \\
	\vspace{5mm}
 We have
\begin{equation}
	\begin{aligned}
		\sin\theta \partial_{t}f(x,t,\phi,\theta) & + \textcolor{red}{\partial_x (\cos\phi \cos\theta \sin^2\theta f)} \\
		& + \textcolor{blue}{\partial_\theta \left(w_x \sin^3 \theta \cos \phi f\right)}
		= \textcolor{blue}{D_{r} \left( \partial_\phi \left(\frac{1}{\sin \theta} \partial_\phi f \right) + \partial_\theta (\sin \theta \partial_\theta f) \right)}, \\
		& Re \partial_{t}w(x,t) = \partial_{xx}w + \delta \left( \bar{\rho} - \int_{0}^{2\pi} \int_{0}^{\pi} f \sin \theta \, d\theta \, d\phi \right). \label{SmochEq_wx}
	\end{aligned}
\end{equation}
\begin{figure}[H]
    \flushleft
	\includegraphics[scale=0.5]{Bilder/Ausrichtung_Partikeln}
\end{figure}
\end{frame}










\section{Hierarchy of Moment Equations for Shear Flow}
\begin{frame}{Ansatz for Derivation of Moment Equations}
	\scriptsize
    Consider approximation of the form
	\begin{align}
		\textcolor{cyan}{f(\textbf{x}, t, \phi, \theta) \approx f^N(\textbf{x},t,\phi,\theta) :=  \sum_{n=0}^{N} \sum_{i=-2n}^{2n} c^i_{2n}(\textbf{x},t) \cdot P^i_{2n}(\phi, \theta)}, \label{spectralmethod}
	\end{align}
	where $P^i_{2n}(\phi, \theta)$, $n = 0, \ldots, N$, $i = -2n, \ldots, 2n$
	\begin{itemize}
		\item are harmonic polynomial basis functions, i.e., the eigenfunctions of the Laplace-Beltrami operator with the eigenvalue $-2n(2n+1)$
		\pause
		\item form an orthonormal basis, wrt. the $L_2$-inner product on the sphere
		\begin{align*}
			(g,h)_{S^2} := \int_{0}^{2\pi} \int_{0}^{\pi} g(\phi, \theta) h(\phi, \theta) \cdot \sin(\theta) d\theta d\phi,
		\end{align*}
	\pause
	\item every square integrable function on $S^2$ can be expressed as a linear combination of spherical harmonics $f(\phi, \theta) = \sum^{\infty}_{n=0} \sum_{i=-n}^{n} c^i_{n} \cdot P^i_{n}(\phi, \theta)$.
	\end{itemize}
\end{frame}

\begin{frame}{Derivation of Moment Equations}
	\scriptsize
	We derive the moment equations by
	\begin{itemize}
		\item Insert ansatz $f^N =  \sum_{n=0}^{N} \sum_{i=-2n}^{2n} c^i_{2n}(x,t) \cdot P^i_{2n}(\phi, \theta)$ into kinetic equation
\begin{align*}
\sin\theta \partial_{t}f^N(x,t,\phi,\theta) + &  \textcolor{red}{\partial_x (\cos\phi \cos\theta \sin^2\theta f^N)} \\
	& 
	 =- \textcolor{blue}{\partial_\theta \left(w_x \sin^3 \theta \cos \phi f^N\right)} +\textcolor{blue}{D_{r} \left( \partial_\phi \left(\frac{1}{\sin \theta} \partial_\phi f^N \right) + \partial_\theta (\sin \theta \partial_\theta f^N) \right)}
\end{align*}
		\item Multiply consecutively with all basis functions used in the ansatz and integrate the resulting equations over the sphere
	\end{itemize}
\pause
The system of moment equations has the general form
\begin{align}
	\partial_t Q + \textcolor{red}{A\partial_x Q} = \textcolor{blue}{D(w_x)Q}+ \textcolor{blue}{D_rE Q},
\end{align}
where $Q=(c^0_0(x,t), c^{-2}_2(x,t), \ldots, c^{2N}_{2N}(x,t))^T$ represents the vector of the moments and \\
\vspace{2mm}
$A,D,E \in \mathbb{R}^{(N+1)(2N+1)x(N+1)(2N+1)}$.
\end{frame}

\begin{frame}{Derivation: A Closer Look}
\scriptsize
Consider 
\begin{align*}
	\textcolor{cyan}{\underbrace{\sin\theta \partial_{t}f^N(x,t,\phi,\theta)}_{[1]}} + &  \underbrace{\partial_x (\cos\phi \cos\theta \sin^2\theta f^N)}_{[2]} \\
	& 
	= -\partial_\theta \left(w_x \sin^3 \theta \cos \phi f^N\right) + D_{r} \left(\partial_\phi \left(\frac{1}{\sin \theta} \partial_\phi f^N \right) + \partial_\theta (\sin \theta \partial_\theta f^N) \right)
\end{align*}
\pause
For $k=0, \ldots, N$, $l=-2k, \ldots, 2k$ we obtain for term $[1]$
\begin{align*}
	&\int_{0}^{2\pi} \int_{0}^{\pi} \textcolor{cyan}{\sin\theta \partial_t \left(\sum_{n=0}^{N} \sum_{i=-2n}^{2n} c^i_{2n}(x,t) \cdot P^i_{2n}(\phi, \theta)\right)}P^l_{2k}(\phi, \theta) \\
	&= \sum_{n=0}^{N} \sum_{i=-2n}^{2n} \partial_t c^i_{2n}(x,t) \int_{0}^{2\pi} \int_{0}^{\pi} \sin\theta P^i_{2n}(\phi, \theta) P^l_{2k}(\phi, \theta) d\phi d\theta \\
	&=  \sum_{n=0}^{N} \sum_{i=-2n}^{2n} \partial_t c^i_{2n}(x,t) (P^i_{2n}(\phi, \theta) P^l_{2k}(\phi, \theta))_{S^2} =  \sum_{n=0}^{N} \sum_{i=-2n}^{2n} \partial_t c^i_{2n}(x,t) \cdot \delta_{n,k} \delta_{i,l} \\
	&= \partial_t c^l_{2k}(x,t)
\end{align*}
\pause
This corresponds to this term
$\textcolor{cyan}{\partial_t Q} + A\partial_x Q = D(w_x)Q+ D_rE Q$.
\end{frame}

\begin{frame}
	\scriptsize
Consider term [2]
\begin{align*}
	\underbrace{\sin\theta \partial_{t}f^N(x,t,\phi,\theta)}_{[1]} + &  \textcolor{cyan}{\underbrace{\partial_x (\cos\phi \cos\theta \sin^2\theta f^N)}_{[2]}} \\
	& 
	= -\partial_\theta \left(w_x \sin^3 \theta \cos \phi f^N\right) + D_{r} \left(\partial_\phi \left(\frac{1}{\sin \theta} \partial_\phi f^N \right) + \partial_\theta (\sin \theta \partial_\theta f^N) \right)
\end{align*}
\pause
This represents the term
$\partial_t Q + \textcolor{cyan}{A\partial_x Q} =  D(w_x)Q+ D_rEQ$ \\
\vspace{2mm}
For $N=1$ the matrix $A$ has the form
\begin{figure}[H]
		\includegraphics[scale=0.5]{Bilder/MatrixA}
\end{figure}
\end{frame}


\begin{comment}
	For $N=1$ the matrix $A$ has the form
	\begin{equation}
		\left[\begin{array}{c:c}
			A_{0,0} & A_{0,1} \\
			\hdashline A_{0,1}^T & A_{1,1}
		\end{array}\right]=\left[\begin{array}{c:ccccc}
			0 & 0 & \frac{1}{\sqrt{15}} & 0 & 0 & 0 \\
			\hdashline 0 & 0 & \frac{1}{7} & 0 & 0 & 0 \\
			\frac{1}{\sqrt{15}} & \frac{1}{7} & 0 & \frac{\sqrt{3}}{21} & 0 & 0 \\
			0 & 0 & \frac{\sqrt{3}}{21} & 0 & 0 & 0 \\
			0 & 0 & 0 & 0 & 0 & \frac{1}{7} \\
			0 & 0 & 0 & 0 & \frac{1}{7} & 0
		\end{array}\right] .
	\end{equation}
\end{comment}

%%%%%%%%%%%%%%%%%%%%%%%%%%%%%%%%%%%%%%%%%%%%%%%%%%%%%%%%%%%%%%%%%
%%%%%%%%%%%%%%%%% Struktur Matrix A N=2 machen %%%%%%%%%%%%%%%%%%
%%%%%%%%%%%%%%%%%%%%%%%%%%%%%%%%%%%%%%%%%%%%%%%%%%%%%%%%%%%%%%%%%


\begin{frame}
	\scriptsize
	For N = 2 the symmetric matrix A has the structure
\begin{figure}[H]
	\includegraphics[scale=0.5]{Bilder/MatrixAN=2}
\end{figure}
\textcolor{cyan}{For any N the system of moment equations is hyperbolic.}
\end{frame}

\begin{frame}
		\scriptsize
Now consider the remaining terms:
	\begin{align*}
		\sin\theta \partial_{t}f^N(x,t,\phi,\theta) &+ \partial_x (\cos\phi \cos\theta \sin^2\theta f^N) \\
		&= \textcolor{cyan}{\underbrace{-\partial_\theta \left(w_x \sin^3 \theta \cos \phi f^N\right)}_{[3]} + \underbrace{D_{r} \left(\partial_\phi \left(\frac{1}{\sin \theta} \partial_\phi f^N \right) + \partial_\theta (\sin \theta \partial_\theta f^N) \right)}_{[4]}}.
	\end{align*}
\pause
\begin{itemize}
	\item Term $[4]$ corresponds to the Laplace-Beltrami operator,  resulting in $\textcolor{cyan}{D_r E Q}$, where the matrix $E$ is a diagonal matrix with the Laplace-Beltrami eigenvalues. 
	\pause
	\item Term $[3]$: We apply the same approach by inserting the ansatz for $f$ and projecting onto the polynomials, resulting in $\textcolor{cyan}{D(w_x)Q}$.
\end{itemize}
\vspace{5mm}
The system of moment equations:
$\partial_t Q + A\partial_x Q =  D(w_x)Q+ D_rEQ$. 
\end{frame}

\begin{comment}
\begin{frame}{Hierarchy of Moment Equations for Shear Flow}
\scriptsize
For $N=1$ we obtain 
\begin{equation}
	\begin{aligned}
		&\partial_t Q + \textcolor{red}{
			\begin{pmatrix}
				0 & 0 & \frac{\sqrt{15}}{15} & 0 & 0 & 0 \\
				0 & 0 & \frac{1}{7} & 0 & 0 & 0 \\
				\frac{\sqrt{15}}{15} & \frac{1}{7} & 0 & \frac{\sqrt{3}}{21} & 0 & 0 \\
				0 & 0 & \frac{\sqrt{3}}{21} & 0 & 0 & 0 \\
				0 & 0 & 0 & 0 & 0 & \frac{1}{7} \\
				0 & 0 & 0 & 0 & \frac{1}{7} & 0
		\end{pmatrix} \cdot \partial_x Q} \\
		&= \textcolor{blue}{
			\begin{pmatrix}
				0 & 0 & 0 & 0 & 0 & 0 \\
				0 & -6D_r & \frac{2}{7}w_x & 0 & 0 & 0 \\
				-\frac{\sqrt{15}}{5}w_x & -\frac{5}{7}w_x & -6D_r & \frac{3\sqrt{3}}{7}w_x & 0 & 0 \\
				0 & 0 & -\frac{4\sqrt{3}}{7}w_x & -6D_r & 0 & 0 \\
				0 & 0 & 0 & 0 & -6D_r & -\frac{5}{7} w_x \\
				0 & 0 & 0 & 0 & \frac{2}{7}w_x & -6D_r
			\end{pmatrix} Q }.
	\end{aligned}
\end{equation}
\end{frame}
\end{comment}




\section{Numerical Simulations}
\begin{frame}{Example: One-Dimensional Shear Flow}
	\scriptsize
Consider
	\begin{align}
		\partial_t Q(x,t) + \textcolor{blue}{A}\partial_x Q(x,t) = \textcolor{red}{\varphi} (Q(x,t)), \label{conservationlaws}
	\end{align}
	where $Q(x,t) = (f_0, c^{-2}_2, c^{-1}_2, ..., c^2_2)$ is. \\
	\vspace{0.5cm}
	Solve this problem by splitting it into two subproblems of the form
	\begin{enumerate}
		\item $\partial_t Q(x,t)  = \varphi (Q(x,t))$
		\item $\partial_t Q(x,t) + A Q_x=0$.
	\end{enumerate}
\end{frame}

\begin{frame}
	\scriptsize
	\begin{enumerate}
		\item Solve the drift diffusion equation
		\begin{equation*}
			\partial_t \left(\begin{array}{c}
				f_0 \\
				c_2^{-2} \\
				c_2^{-1} \\
				c_2^0 \\
				c_2^1 \\
				c_2^2
			\end{array}\right)  = \begin{pmatrix}
				0 & 0 & 0 & 0 & 0 & 0 \\
				0 & -6D_r & \frac{2}{7}w_x & 0 & 0 & 0 \\
				-\frac{\sqrt{15}}{5}w_x & -\frac{5}{7}w_x & -6D_r & \frac{3\sqrt{3}}{7}w_x & 0 & 0 \\
				0 & 0 & -\frac{4\sqrt{3}}{7}w_x & -6D_r & 0 & 0 \\
				0 & 0 & 0 & 0 & -6D_r & -\frac{5}{7} w_x\\
				0 & 0 & 0 & 0 & \frac{2}{7}w_x & -6D_r
			\end{pmatrix} \cdot
			\left(\begin{array}{c}
				0 \\
				c^{-2}_2(t) \\
				c_2^{-1}(t) \\
				c_2^0(t) \\
				c_2^1(t) \\
				c_2^2(t)
			\end{array}\right).
		\end{equation*}
		with Runge-Kutta 4th order method .
		\item Solve the advection equation
		$$
		\partial_t \left(\begin{array}{c}
			f_0 \\
			c_2^{-2} \\
			c_2^{-1} \\
			c_2^0 \\
			c_2^1 \\
			c_2^2
		\end{array}\right) + \begin{pmatrix}
			0 & 0 & \frac{\sqrt{15}}{15} & 0 & 0 & 0 \\
			0 & 0 & \frac{1}{7} & 0 & 0 & 0 \\
			\frac{\sqrt{15}}{15} & \frac{1}{7} & 0 & \frac{\sqrt{3}}{21} & 0 &  0 \\
			0 & 0 & \frac{\sqrt{3}}{21} & 0 & 0 & 0 \\
			0 & 0 & 0 & 0 & 0 & \frac{1}{7}\\
			0 & 0 & 0 & 0 & \frac{1}{7} & 0
		\end{pmatrix} \partial_x \left(\begin{array}{c}
			f_0 \\
			c_2^{-2} \\
			c_2^{-1} \\
			c_2^0 \\
			c_2^1 \\
			c_2^2
		\end{array}\right) = 0
		$$
		with one-dimensional resolution Wave Propagation Algorithm.
	\end{enumerate}
\end{frame}

\begin{frame}{Macroscopic Transport}
	\scriptsize
	Consider a Riemann Problem for externally imposed shear flow with $wx/D_r = 0.5$ for the left state and $wx/D_r = 1$ for the right state.
	\begin{figure}[H]
		\centering
		\begin{minipage}{0.32\textwidth}
			\includegraphics[width=\textwidth]{Bilder_wx/Wavepropa/red=12th_blue=2nd_wx=1_leftDr1_rightDr2_Awp12th}
		\end{minipage}
		\hfill
		\begin{minipage}{0.32\textwidth}
			\includegraphics[width=\textwidth]{Bilder_wx/Wavepropa/red=12th_blue=4th_wx=1_leftDr1_rightDr2_Awp12th}
		\end{minipage}
		\hfill
		\begin{minipage}{0.32\textwidth}
			\includegraphics[width=\textwidth]{Bilder_wx/Wavepropa/red=12th_blue=6th_wx=1_leftDr1_rightDr2_Awp12th}
		\end{minipage}
		\vfill
		\begin{minipage}{0.32\textwidth}
			\includegraphics[width=\textwidth]{Bilder_wx/Wavepropa/red=12th_blue=8th_wx=1_leftDr1_rightDr2_Awp12th}
		\end{minipage}
		\hfill
		\begin{minipage}{0.32\textwidth}
			\includegraphics[width=\textwidth]{Bilder_wx/Wavepropa/red=12th_blue=10th_wx=1_leftDr1_rightDr2_Awp12th}
		\end{minipage}
		\hfill
		\begin{minipage}{0.3\textwidth}
			% Optional: Leave empty or add another image
		\end{minipage}
		\caption{Solution of the Riemann problem for $f_0$ at the time $t = 1.875$ using the moment system for different $N$ (blue curve) in comparison with $N = 6$ (red curve).}
	\end{figure}
\end{frame}

\begin{frame}
	\scriptsize
	\textbf{Setting:}\\
	\begin{itemize}
		\item $w_{xi} = \pi \cdot \cos(\pi\cdot x_i)$
		\item Initial value $q(i,1)$ is set to $\frac{1}{2\sqrt{\pi}}$, rest $0$
		\item grid cells in x direction is set to $1000$
	\end{itemize}
	\begin{figure}[H]
		\centering
		\begin{minipage}{0.32\textwidth}
			\includegraphics[width=\textwidth]{Bilder_wx/Wavepropa/red=12th_blue=2nd_wx=sin(pix)_Awp=0.5sqrt(pi)}
		\end{minipage}
		\hfill
		\begin{minipage}{0.32\textwidth}
			\includegraphics[width=\textwidth]{Bilder_wx/Wavepropa/red=12th_blue=4th_wx=sin(pix)_Awp=0.5sqrt(pi)}
		\end{minipage}
		\hfill
		\begin{minipage}{0.32\textwidth}
			\includegraphics[width=\textwidth]{Bilder_wx/Wavepropa/red=12th_blue=10th_wx=sin(pix)_Awp=0.5sqrt(pi)}
		\end{minipage}
		\caption{Solution of the problem (\ref{conservationlaws}) for $f_0$ at the time $t = 3$ using the moment system for different $N$ (blue curve) in comparison with $N = 6$ (red curve).}
	\end{figure}
\end{frame}

\begin{frame}{Simulation of One-Dimensional Coupled Shear Flow Problem}
\scriptsize
Consider the coupled moment system
\begin{align*}
&\partial_t Q(\boldsymbol{x},t) + A\partial_x Q(\boldsymbol{x},t) =\varphi (Q(\boldsymbol{x},t)) \\
	&Re\partial_{x}w = \partial_{xx}w + \delta(\bar{\rho}-\rho(x,t)).
\end{align*}
on the interval $[0, 100]$ with periodic boundary conditions and initial data
\begin{align*}
	&\rho(x,0) = exp(-10(x-50)^2), \\
	&w(x,0) = 1 ,\\
	&D_r = 0.01, \delta = 1
\end{align*}
\end{frame}
\begin{frame}{Conclusion}
	Inhalt...
\end{frame}

\begin{frame}{Appendix}
		\begin{table}[H]
			\scriptsize
			\begin{tabular}{|c|c|}
				\hline
				0th order	& 2nd order \\
				\hline
				& $P^{-2}_2 = \sqrt{\frac{15}{16\pi}}\sin^2(\theta)\cos(2\phi)$ \\
				& $P^{-1}_2 = \sqrt{\frac{15}{4\pi}}\sin(\theta)\cos(\theta)\cos(\phi)$ \\
				$P^0_0 = \sqrt{\frac{1}{4\pi}} \cdot 1$	& $P^0_2 = \sqrt{\frac{45}{16\pi}}\cos^2(\theta) - \frac{1}{3}$  \\
				&  $P^1_2 = \sqrt{\frac{15}{4\pi}}\sin(\theta)\cos(\theta)\sin(\phi)$\\
				&  $P^2_2 = \sqrt{\frac{15}{16\pi}}\sin^2(\theta)\sin(2\phi)$\\
				\hline
			\end{tabular}
			\caption{Normalized harmonic polynomial basis functions}
		\end{table}
\end{frame}


\begin{comment}
\begin{frame}
	\centering
	Thank you for listening!
\end{frame}
\end{comment}

\AtBeginSection[]
%{
% \begin{frame}<beamer>
% \frametitle{Table of Contents}
% \tableofcontents[currentsection]
% \end{frame}
%}



%Ende
%\begin{frame}{Ende}
%	\centering
%	Thank you for listening!
%\end{frame}


\begin{frame}[allowframebreaks]
  \frametitle{Literatur}
  \nocite{*} %es werden auch Quellen aufgefuehrt, die nicht in den Folien zitiert sind.
  \bibliography{Bibliothek}
  \bibliographystyle{abbrv}
\end{frame}


\end{document}