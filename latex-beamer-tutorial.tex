\documentclass[lightblue, notheorems, xcolor=dvipsnames]{beamer}
%\usepackage{beamerthemeshadow}
\usetheme{Madrid}

\usepackage[english]{babel}
\usepackage[utf8]{inputenc}
\usepackage[T1]{fontenc}
\usepackage{amsmath, amssymb, amsthm, amsfonts} %einige hilfreiche Mathe-Pakete
\setbeamertemplate{theorems}[ams style]
\usepackage{tabularx, multirow} %fuer Tabellen
\usepackage{graphicx, tikz} %Grafiken
\usepackage{setspace} %Paket f\"ur das Formatieren von Quellen unten auf einer Folie
\newcommand{\diff}{\mathop{}\!\mathrm{d}}
\usepackage[T1]{fontenc}

% Algorithm
\usepackage[ruled,vlined]{algorithm2e}
\usepackage{algpseudocode}
\usepackage{algorithmicx}



% BIlder
\setbeamertemplate{caption}[numbered]
\usepackage{subcaption}
%\usepackage{subfig}
\usepackage{graphicx}
\usepackage{float}

\usepackage{comment}

% Ausblenden Navigation
\beamertemplatenavigationsymbolsempty

% definiere Enviroments
\theoremstyle{definition}
\newtheorem{defi}{Definition}[section]
\newtheorem{idea}{Idee}[section]
\theoremstyle{plain}
\newtheorem{theo}[defi]{Satz}
\newtheorem{coroll}[defi]{Korollar}%
\newtheorem{lem}[defi]{Lemma}%
\newtheorem{aim}[defi]{Ziel}
%\newtheorem*{proof}{Beweis}%[theorem]

\theoremstyle{example}
\newtheorem{remark}[defi]{Bemerkung}%
\newtheorem{examp}[defi]{Beispiel}%
\newtheorem*{lsg}{Lösung}


\DeclareMathOperator*{\argmax}{argmax}


\iffalse
\expandafter\def\expandafter\insertshorttitle\expandafter{%
    \insertshorttitle\hfill%
    \insertframenumber\,/\,\inserttotalframenumber}
\fi



\SetKwRepeat{Do}{do}{while}%
\begin{document}
   %\maketitle  %erstellt Titelseite mit oben stehenden Informationen
  % optional: mit 
  \author{Bella My Phuong Quynh Duong}
  \date{September 26, 2023} %Datum des Vortrags eintragen
  \institute{Heinrich-Heine University D\"usseldorf}
  \title[Structure-Preserving Multi-Scale Methods
  for Complex Fluids]{Numerical Solution with Spectral Method for Smoluchowski Equation on $S^2$}
  
  \begin{frame}
  	\titlepage
  	\begin{figure}[htpb]
  		\begin{center}
  			\includegraphics[width=0.2\textwidth]{logo.png}
  		\end{center}
  	\end{figure}
  \end{frame}
\begin{frame}
   \frametitle{Table of Contents}
\tableofcontents %dieser Befehl erzeugt das Inhaltsverzeichnis
\end{frame}

\AtBeginSection[]
{
	\begin{frame}<beamer>
		\frametitle{Overview}
		\tableofcontents[currentsection]
	\end{frame}
}

\section{Introduction}

\begin{frame}{Mathematical Model for the Sedimentation of Rod-Like Particles}
	\scriptsize
\text{Coupling of a kinetic Smoluchowski equation with Navier-Stokes equation}
\begin{align*}
\partial_t f+ \nabla_{\boldsymbol{x}} \cdot(\boldsymbol{u} f) & +\nabla_{\boldsymbol{n}} \cdot\left(P_{\boldsymbol{n}^{\perp}} \nabla_{\boldsymbol{x}} \boldsymbol{u} \boldsymbol{n} f\right)- \nabla_{\boldsymbol{x}} \cdot\left((I+\boldsymbol{n} \otimes \boldsymbol{n}) \boldsymbol{e_3} f\right) \\
	& =D_r \Delta_n f, \\
	\operatorname{Re}\left(\partial_t \boldsymbol{u}+\left(\boldsymbol{u} \cdot \nabla_{\boldsymbol{x}}\right) \boldsymbol{u}\right) & =\Delta_{\boldsymbol{x}} \boldsymbol{u}-\nabla_{\boldsymbol{x}} p-\delta \left(\int_{S^{d-1}} f d \boldsymbol{n} \right) \boldsymbol{e_3}, \\
	\nabla_{\boldsymbol{x}} \cdot \boldsymbol{u} & =0,
\end{align*}
where $f = f(\boldsymbol{x}, t, \boldsymbol{n})$ represents the particle distribution of rod-like particles as a function of time $t$, space $\boldsymbol{x} \in \mathbb{R}^3$ and orientation $\boldsymbol{n} \in  S^2$. 
$D_r, \delta$ and $Re$ are non-dimensional parameters.
\begin{beamercolorbox}[sep=1em,wd=\linewidth,right]{}
	\tiny{Helzel $\&$ Tzavaras, 2017}
\end{beamercolorbox}
\end{frame}

\begin{frame}{Outline of the Project}
	\scriptsize
	\begin{block}{Goal}
		\begin{itemize}
			\item Reduce the high-dimensional kinetic equation (in space and orientation) to a lower-dimensional system of moment equations (in space).
			\item Derive and approximate hierarchies of moment equations for the coupled kinetic-fluid model with \textcolor{cyan}{$f$ on $S^2$}
		\end{itemize}
	\end{block}
    \pause
	\begin{block}{Our Approach}
		 Investigate different coupled flow situations
		\begin{itemize}
			\item externally imposed velocity field
            \item coupled problems: 1D shear flow, 2D rectilinear flow and
             \pause now 3D flow with periodic boundary conditions
		\end{itemize}
    \end{block}
\end{frame}

%----------------------------------------------------------
%----------------------------------------------------------


\begin{comment}
\begin{frame}
	\scriptsize
	\begin{figure}[H]
		\centering
		\begin{minipage}{0.4\textwidth}
			\includegraphics[scale=0.3]{Bilder_wxwy/t=0_wxwy=1_wxwy=-1}
		\end{minipage}
		\hfill 
		\begin{minipage}{0.4\textwidth}
			\includegraphics[scale=0.3]{Bilder_wxwy/t=200_wxwy=1_wxwy=-1}
		\end{minipage}
		\caption{Numerical results for $c^0_0$ at different times using $D_r=1$ and $w_x=w_y=1$ for $x<50$ and $w_x=w_y=-1$ otherwise. A cluster with higher particle density splits into two, each moving in opposite directions.}
		
	\begin{minipage}{0.4\textwidth}
		\centering
		\includegraphics[scale=0.25]{Bilder_wxwy/14th_t=100_mx=my=200_random_Dr=1_(1.d0+(1.d-2rand(0)-5.d-4))Divide(2.d0dsqrt(pi))}
		\end{minipage}
		\hfill 
		\begin{minipage}{0.4\textwidth}
			\centering
			\includegraphics[scale=0.25]{Bilder_wxwy/14th_t=200_mx=my=200_random_Dr=1_(1.d0+(1.d-2rand(0)-5.d-4))Divide(2.d0dsqrt(pi))}
		\end{minipage}
		\caption{Solution structure of \(c^0_0\) with \(N = 7\).}
	\end{figure}
\end{frame}
\end{comment}










%\section{Theoretical background}
\begin{frame}\frametitle{Spherical Harmonics}
	\small
		\begin{theorem}
			Let $P^{i_1}_{n_1}(\phi, \theta)$ and $P^{i_2}_{n_2}(\phi, \theta)$ are two different harmonic polynomial basis functions. Then the integral over the sphere is equal to zero
			\begin{align*}
				\int_{0}^{2\pi} \int_{0}^{\pi} P^{i_1}_{n_1} \cdot P^{i_2}_{n_2} \cdot \sin(\theta) d\theta d\phi = 0.
			\end{align*}
		\end{theorem}

		\begin{theorem}
			Let $P^i_{2n}$ with $n = 0, ..., \infty$ and $i = n, ..., -n$ be the normalized harmonic polynomial basis functions. The integral over $S^2$ of the normalized basis functions is equal to one.
		\end{theorem}

	The properties show that the spherical harmonics on the unit sphere form an orthonormal basis, which means that they are orthogonal to each other, and the integral over the unit sphere of each individual spherical harmonic is equal to 1 when they are normalized.
\end{frame}

\begin{frame}
	\begin{definition}
		The \textbf{Laplace-Beltrami operator} on the unit sphere $S^2$ is given by
		\begin{align*}
			\Delta_{S^2} f = \frac{1}{\sin ^2 \theta} \partial_{\phi \phi} f + \frac{1}{\sin \theta} \partial_\theta\left(\sin \theta \partial_\theta f\right).
		\end{align*}
		The spherical harmonic function are the eigenfunctions of Laplace-Beltrami operator with the eigenvalues $(- \ell (\ell +1),\ell \in \mathbb{N}_0)$(see \cite{zbMATH01218597})
		\begin{align*}
			\Delta_{S^2}P^i_n = -\ell(\ell+1) P^i_n,
		\end{align*}
		with $n = 0, ..., \infty$ and $i = n ,..., -n$.
	\end{definition}
\end{frame}

%\begin{columns}[T, totalwidth=\textwidth]
%	\begin{column}{0.44\textwidth}
%		\begin{definition}
%			\begin{itemize}
%				\item x: Input image
%				\item z: Internal latent vector
%				\item $\phi$:  Encoder model parameter
%				\item $\theta$:  Decoder model parameter
%			\end{itemize}
%		\end{definition}
%	\end{column}
%	\begin{column}{0.53\textwidth}
%		\begin{definition}
%			\begin{itemize}
%				\item $q_{\phi}(z|x)$: Probability distr. (Encoder)
%				\item $p_{\theta}(x|z)$: Probability distr. (Decoder)
%				\item $KL(P,Q) = \sum_{x \in X}^{} P(x) log(\frac{P(x)}{Q(x)})$	
%			\end{itemize}
%		\end{definition}
%	\end{column}
%\end{columns}
\section{Spectral method}
\begin{frame}{Mathematical Model for the Sedimentation of Rod-Like Particles \cite{zbMATH06724175}}
	\scriptsize
	\text{Coupling of a Smoluchowski Equation and a Navier-Stokes Equation}
	\begin{align*}
		\partial_t f+\nabla_{\boldsymbol{x}} \cdot(\boldsymbol{u} f) & +\nabla_{\boldsymbol{n}} \cdot\left(P_{\boldsymbol{n}^{\perp}} \nabla_{\boldsymbol{x}} \boldsymbol{u} \boldsymbol{n} f\right)-\nabla_{\boldsymbol{x}} \cdot\left((I+\boldsymbol{n} \otimes \boldsymbol{n}) e_3 f\right) \\
		& =D_r \Delta_n f+\gamma \nabla_{\boldsymbol{x}} \cdot(I+\boldsymbol{n} \otimes \boldsymbol{n}) \nabla_{\boldsymbol{x}} f, \notag \\
		\sigma & =\int_{S^{d-1}}(\text{d} \; \boldsymbol{n} \otimes \boldsymbol{n}-I) f d \boldsymbol{n},  \\
		\operatorname{Re}\left(\partial_t \boldsymbol{u}+\left(\boldsymbol{u} \cdot \nabla_{\boldsymbol{x}}\right) \boldsymbol{u}\right) & =\Delta_{\boldsymbol{x}} \boldsymbol{u}-\nabla_{\boldsymbol{x}} p+\delta \gamma \nabla_{\boldsymbol{x}} \cdot \sigma-\delta \int_{S^{d-1}} f d \boldsymbol{n} \, e_3, \\
		\nabla_{\boldsymbol{x}} \cdot \boldsymbol{u} & =0,
	\end{align*}
	where $f = f(t; x; n), x \in \mathbb{R}^d , n \in  S^{d-1}, t \in \mathbb{R}$ is a density distribution function of particle orientation. $D_r, \gamma, \delta$ and $Re$ are non-dimensional values.
\end{frame}


\begin{frame}
	\scriptsize
	Consider 
	\begin{align}
		\textcolor{blue}{\partial_t f}+ \nabla_{\boldsymbol{x}} \cdot(\boldsymbol{u} f) & +\textcolor{blue}{\nabla_{\boldsymbol{n}} \cdot\left(P_{\boldsymbol{n}^{\perp}} \nabla_{\boldsymbol{x}} \boldsymbol{u} \boldsymbol{n} f\right)}-\nabla_{\boldsymbol{x}} \cdot\left((I+\boldsymbol{n} \otimes \boldsymbol{n}) e_3 f\right) \notag \\
		& = \textcolor{blue}{D_r \Delta_n f}+\gamma \nabla_{\boldsymbol{x}} \cdot(I+\boldsymbol{n} \otimes \boldsymbol{n}) \nabla_{\boldsymbol{x}} f. \label{SmochEq_kurz}
	\end{align}
	Rewrite the equation (\ref{SmochEq_kurz}) in spherical coordinates (\cite{zbMATH05037679}, \cite{zbMATH07295185}) and it follows
	\begin{align}
		\sin \theta \partial_t f + \partial_\phi\left(a(\phi, \theta) f\right)+\partial_\theta\left(b(\phi, \theta) f\right) = D_r \left(\partial_\phi\left(\frac{1}{\sin \theta} \partial_\phi f\right)+\partial_\theta\left(\sin \theta \partial_\theta f\right)\right), \label{Smochluch_S2}
	\end{align}
	with $\phi \in [0, 2 \cdot \pi]$ and $\theta \in [0, \pi]$.
	Solving the Smoluchoswki equation (\ref{Smochluch_S2}) on $S^2$ by using a \textcolor{cyan}{spectral method}, which is based on the ansatz
	\begin{align}
		f(\phi, \theta, t) = f_0(t) \cdot P_0^0 + \sum_{n=1}^{\infty} \sum_{i=-n}^{n} c^i_{2n}(t) \cdot P^i_{2n}(\phi, \theta), \label{ansatz}
	\end{align}
	where $P^i_{2n}(\phi, \theta)$ are harmonic polynomial basis functions. %, i.e. are eigenfunctions of Laplace Beltrami operator.
\end{frame}

%%%%%%%%%%%%%%%%%%%%%%
% Properties
%%%%%%%%%%%%%%%%%%%%%%

\begin{frame}{Harmonic polynomial basis functions}
	\scriptsize
	Let ${P}^{i}_{n}(\phi, \theta)$ with $n = 0, ..., \infty$ and $i = n, ..., -n$ be the basis function
	\begin{align*}
		TODO
	\end{align*}
	The scalar product of any two basis functions over sphere is defined as follows
	\begin{align*}
		<P^i_n, P^l_m>_{S^2} = \int_{0}^{2\pi} \int_{0}^{\pi} P^{i}_{n}(\phi, \theta) \cdot P^{l}_{m}(\phi, \theta) \cdot \sin(\theta) d\theta d\phi.
	\end{align*}
\end{frame}

\begin{frame}{Properties of harmonic polynomial basis functions}
	\scriptsize
	\begin{block}{Property 1}
		Let $P^{i}_{n}(\phi, \theta)$ and $P^{l}_{m}(\phi, \theta)$ are two different harmonic polynomial basis functions. Then
		\begin{align*}
			<P^i_n, P^l_m>_{S^2} = 0,
		\end{align*}
	for $i \neq l$ or $n \neq m$.
	\end{block}
	
	\begin{block}{Property 2}
		Let $P^i_{n}$ be the normalized harmonic polynomial basis functions. Then
		\begin{align*}
			<P^i_n, P^i_n>_{S^2} = 1.
		\end{align*}
	\end{block}

	\begin{block}{Property 3}
		The spherical harmonic function are the eigenfunctions of Laplace-Beltrami operator with the eigenvalues $(- n (n +1),n \in \mathbb{N}_0)$(see \cite{zbMATH01218597})
		\begin{align*}
			\Delta_{S^2}P^i_n = -n(n+1) P^i_n.
		\end{align*}
	\end{block}
\end{frame}


%%%%%%%%%%%%%%%%%%%%%%%%%%%%%%%%%%%%%%
% Back to the spectral method
%%%%%%%%%%%%%%%%%%%%%%%%%%%%%%%%%%%%%%

\begin{frame}{Spectral method}
	\centering
	\scriptsize
	Recall the Smochluchowski equation on $S^2$
	\begin{align}
		\underbrace{\sin \theta \partial_t f}_{(1)} + \underbrace{\partial_\phi\left(a(\phi, \theta) f\right)+\partial_\theta\left(b(\phi, \theta) f\right)}_{(2)} = \underbrace{D_r \left(\partial_\phi\left(\frac{1}{\sin \theta} \partial_\phi f\right)+\partial_\theta\left(\sin \theta \partial_\theta f\right)\right)}_{(3)} \label{Smoch_S2}
	\end{align}	
	and the ansatz
	\begin{align}
		f(\phi, \theta, t) = f_0(t) \cdot P_0^0 + \sum_{n=1}^{\infty} \sum_{i=-n}^{n} c^i_{2n}(t) \cdot P^i_{2n}(\phi, \theta).
	\end{align}
\end{frame}
\begin{frame}
	\scriptsize
	
	The Laplace Beltrami operator (\cite{zbMATH07295185}) on the unit sphere $S^2$ is given by
	\begin{align}
		\Delta_{S^2} f = \frac{1}{\sin ^2 \theta} \partial_{\phi \phi} f + \frac{1}{\sin \theta} \partial_\theta\left(\sin \theta \partial_\theta f\right). \label{laplace_eq}
	\end{align}
	From property 3 and the equation $(\ref{laplace_eq})$, it follows for the term (3)
	\begin{align*}
		\Delta_{S^2} P^i_{2n} = -n(n+1)P^i_{2n}.
	\end{align*}
	For the term $(1)$ and $(2)$, we inserting (5) in (4), multiplying with each basis function and integrate it over $S^2$. We derive a system of ODEs for the coefficients
	\scriptsize
	\begin{equation}
		\left(\begin{array}{c}
			f_0 \\
			c_2^{-2} \\
			\vdots \\
			c_{2n}^i
		\end{array}\right)^{\prime}=A\left(\begin{array}{c}
			f_0 \\
			c_2^{-2} \\
			\vdots \\
			c_{2n}^i
		\end{array}\right),
	\end{equation}
with $A \in \mathbb{R}^{cnxcn}$
\begin{equation*}
	c n= \begin{cases}\text { order }=2: & cn= 2 \cdot \text {order}+2 \\ \text { order }=\text {even: } & c2=6 \\  &c n=c2 \\ &c n=c2+\sum_{i=4}^{order}(2 i+1)\end{cases}
\end{equation*}
\end{frame}

%%%%%%%%%%%%%%%%%%
% Example
%%%%%%%%%%%%%%%%%%
\begin{frame}
	\centering
	Example: Shear flow
\end{frame}

\begin{frame}{Example: Shear flow}
	\scriptsize
	Consider the Smoluchowski equation $(\ref{Smoch_S2})$ with the velocity gradient 
	\begin{align*}
		\vec{u}=\left(\begin{array}{l}
			u(x,y,z) \\
			v(x,y,z) \\
			w(x,y,z)
		\end{array}\right),
		\nabla_x \vec{u}_{\mathrm{ext}}=\left(\begin{array}{lll}
			u_{x} & u_{y} & u_{z} \\
			v_{x} & v_{y} & v_{z} \\
			w_{x} & w_{y} & w_{z}
		\end{array}\right)=\left(\begin{array}{ccc}
			0 & 1 & 0 \\
			0 & 0 & 0 \\
			0 & 0 & 0
		\end{array}\right) .
	\end{align*}
	With the given velocity gradient it follows
	\begin{align}
		\partial_{t}\left(\sin \theta f\right) &+ \partial_\theta\left(\sin \phi \cos \phi \sin ^2 \theta \cos \theta f\right)+ \partial_\phi\left(- \sin \theta \sin ^2 \phi f \right) \nonumber \\
		&=D_{r}\left(\partial_\theta \left(\sin \theta \partial_\theta f\right)+ \partial_\phi\left(\frac{1}{\sin \theta} \partial_\phi f\right)\right). \label{smoEq} 
	\end{align}
	Consider the ansatz with the zeroth order
	\begin{align}
		f(\phi, \theta, t)= f_0(t) \cdot P_0^0 \label{ansatz_0nd}.
	\end{align}

	Insert the ansatz (\ref{ansatz_0nd}) in (\ref{smoEq})
	\begin{align}
		\partial_{t}(f_0(t) \cdot P_0^0)+\frac{1}{\sin \theta}\left(\partial_\theta(\ldots)+\partial_\phi(\ldots)\right)=\frac{1}{\sin \theta} D_r \left(\ldots \right) \label{eq_mitAnsatz}.
	\end{align}
\end{frame}

\begin{frame}
	\scriptsize
	We know
	\begin{align*}
		\Delta_{S^2} P_0^0(\phi, \theta) = \frac{1}{\sin \theta} D_r (\ldots) = \lambda_{2n,i} \cdot P^{0}_0(\phi, \theta),
	\end{align*}
	where $\lambda_{2n,i}$ is the corresponding eigenvalue.\\
	Since $P_0^0(\phi, \theta) = 1$ does not depend on $\phi$ and $\theta$, the partial derivatives will be zero
	\begin{align}
		\Delta_{S^2} P_0^0(\phi, \theta) = 0.
	\end{align}
	Consider the rest of the equation (\ref{eq_mitAnsatz})
	\begin{align*}
		\underbrace{\partial_{t}(f_0(t) \cdot P_0^0)}_{(1)}+ \underbrace{\frac{1}{\sin\theta}\left( \partial_\theta(\sin \phi \cos \phi \sin ^2 \theta \cos \theta \cdot f_0(t) \cdot P_0^0)+ \partial_\phi(- \sin \theta \sin ^2 \phi \cdot f_0(t) \cdot P_0^0)\right)}_{(2)}.
	\end{align*}
	It is
	\begin{align*}
		\partial_t\left(f_0(t) \cdot P^0_0\right)=f_0^{\prime}(t).
	\end{align*}
	Let $z(\phi, \theta) := (2)$
\end{frame}

\begin{frame}
	\scriptsize
	Project the solution $z(\phi, \theta)$ onto all polynomials to find out which polynomial are needed
	\begin{align*}
		\int_{0}^{2\pi} \int_{0}^{\pi} z(\phi, \theta) \cdot P^{-2}_2(\phi, \theta) \, \sin \theta d\theta d\phi &\overset{Maple}{=} 0 \\
		\int_{0}^{2\pi} \int_{0}^{\pi} z(\phi, \theta) \cdot P^{-1}_2(\phi, \theta) \, \sin \theta d\theta d\phi &\overset{Maple}{=} 0 \\
		\int_{0}^{2\pi} \int_{0}^{\pi} z(\phi, \theta) \cdot P^{0}_2(\phi, \theta) \, \sin \theta d\theta d\phi &\overset{Maple}{=} 0 \\
		\int_{0}^{2\pi} \int_{0}^{\pi} z(\phi, \theta) \cdot P^{1}_2(\phi, \theta) \, \sin \theta d\theta d\phi &\overset{Maple}{=} 0 \\
		\int_{0}^{2\pi} \int_{0}^{\pi} z(\phi, \theta) \cdot P^{2}_2(\phi, \theta) \, \sin \theta d\theta d\phi &\overset{Maple}{=} -\frac{\sqrt{15}}{5}
	\end{align*}
\end{frame}

\begin{frame}
	\scriptsize
	It follows
	\begin{align}
		f_0(t) \cdot P^0_0 \cdot \frac{1}{\sin \theta}\left(\partial_\theta\left(\sin \phi \cos \phi \sin ^2 \theta \cos \theta\right)+\partial_\phi \left(-\sin \theta \sin ^2 \phi\right)\right) = f_0(t) \left[ -\frac{\sqrt{15}}{2} P^2_2 \right]. \label{teil1}
	\end{align}
	Together we have
	\begin{align*}
		f_0^{\prime}(t)-\frac{\sqrt{15}}{2} f_0(t) P_2^2(\phi,\theta) = 0\cdot  P^0_0(\phi, \theta)  D_r.
	\end{align*}
\end{frame}

\begin{frame}
	\scriptsize
	\centering
	For the ansatzfunction with higher order, the calculation is done in the same way. \\
	As an example we obtain an ODE system with ansatzfunction of the $2nd.$ order
	\begin{equation}
		\left(\begin{array}{c}
			f_0^{\prime}(t) \\
			c_2^{-2} \\
			c_2^{-1} \\
			c_2^0 \\
			c_2^1 \\
			c_2^2
		\end{array}\right)=\left(\begin{array}{cccccc}
			0 & 0 & 0 & 0 & 0 & 0 \\
			0 & -6 D_r & 0 & 0 & 0 & 1 \\
			0 & 0 & -6 D_r & 0 & 5/7 & 0 \\
			0 & 0 & 0 & -6 D_r & 0 & -\frac{\sqrt{3}}{7} \\
			0 & 0 & -2/7 & 0 & -6 D_r & 0 \\
			\frac{\sqrt{15}}{5} & 1 & 0 & -\frac{\sqrt{3}}{7} & 0 & -6 D_r
		\end{array}\right) \cdot\left(\begin{array}{c}
			f_0 \\
			c_2^{-2} \\
			c_2^{-1} \\
			c_2^0 \\
			c_2^1 \\
			c_2^2
		\end{array}\right)
	\end{equation}
\end{frame}

\begin{comment}
\begin{frame}
	\begin{table}[H]
		\scriptsize
		\begin{tabular}{|c|c|}
			\hline
			0th order	& 2nd order \\
			\hline
			& $P^{-2}_2 = \sqrt{\frac{15}{16\pi}}\sin^2(\theta)\cos(2\phi)$ \\
			& $P^{-1}_2 = \sqrt{\frac{15}{4\pi}}\sin(\theta)\cos(\theta)\cos(\phi)$ \\
			$P^0_0 = \sqrt{\frac{1}{4\pi}} \cdot 1$	& $P^0_2 = \sqrt{\frac{45}{16\pi}}\cos^2(\theta) - \frac{1}{3}$  \\
			&  $P^1_2 = \sqrt{\frac{15}{4\pi}}\sin(\theta)\cos(\theta)\sin(\phi)$\\
			&  $P^2_2 = \sqrt{\frac{15}{16\pi}}\sin^2(\theta)\sin(2\phi)$\\
			\hline
		\end{tabular}
		\caption{Normalized harmonic polynomial basis functions.}
	\end{table}
\end{frame}
\end{comment}

%\begin{frame}
%	\begin{itemize}
%		\item The harmonic polynomial basis function of the second order are the eigenfunctions of the Laplace Beltrami operator with eigenvalue $-6$. 
%		\item $P^{-4}_4, ..., P^4_4$ are eigenfunctions of the Laplace Beltrami operator with eigenvalue $-20$.
		%\item For higher order there exist also an eigenvalue relation the Laplace Beltrami operator.
%	\end{itemize}
%\end{frame}



\section{Numerical Results}
\begin{comment}
\subsection{Shear flow on $S^1$ and $S^2$}

\begin{frame}
	\centering
	Shear flow on $S^1$
\end{frame}

\begin{frame}{Shear flow on $S^1$}
	\scriptsize
	The externally imposed velocity gradient has the form
	\begin{figure}
	\centering
	\begin{equation}
	\nabla_{\mathbf{x}} \mathbf{u}=\begin{pmatrix}
		0 & 1 & 0 \\
		0 & 0 & 0 \\
		0 & 0 & 0
	\end{pmatrix}
	\end{equation}
	\end{figure}

TODO
\end{frame}

\subsubsection{Stability analysis}
\begin{frame}{Shear flow  on $S^1$: Stability analysis}
	\begin{figure}
		\centering
		\begin{minipage}{0.4\linewidth}
			\includegraphics[scale=.42]{Bilder/Stability_analysis_shearflow_N=1_dr0.1}
		\end{minipage}
		\hspace{1cm}
		\begin{minipage}{0.4\linewidth}
			\centering
			\includegraphics[scale=0.42]{Bilder/Stability_analysis_shearflow_N=1_dr1}
		\end{minipage}
		\caption{Eigenvalue of matrix $A$ with $N=1$}
	\end{figure}
	
	\begin{block}{Proposition 2}
		Sprectal method for $N=1$ is stable with for both small and large $D_r$.
	\end{block}
\end{frame}


\begin{frame}{Shear flow  on $S^1$: Stability analysis}
	\begin{figure}
		\centering
		\begin{minipage}{0.4\linewidth}
			\includegraphics[scale=.42]{Bilder/Stability_analysis_shearflow_N=3_dr0.1}
		\end{minipage}
		\hspace{1cm}
		\begin{minipage}{0.4\linewidth}
			\centering
			\includegraphics[scale=0.42]{Bilder/Stability_analysis_shearflow_N=3_dr1}
		\end{minipage}
		\caption{Eigenvalue of matrix $A$ $N=3$}
	\end{figure}
	
	\begin{block}{Proposition 3}
		Sprectal method for $N=3$ is stable with for both small and large $D_r$.
	\end{block}
\end{frame}
	Inhalt...
\end{comment}

%%%%%%%%%%%%%%%%%%%%%%%%%%%%%%%%%%%%%
%%%%%%%%%%%%%%%%%%%%%%%%%%%%%%%%%%%%%
% Shear flow on S^2 
%%%%%%%%%%%%%%%%%%%%%%%%%%%%%%%%%%%%%
%%%%%%%%%%%%%%%%%%%%%%%%%%%%%%%%%%%%%

\subsection{Shear flow on $S^2$}
\begin{frame}
	\centering
	Shear flow on $S^2$
\end{frame}

\begin{frame}{Shear flow on $S^2$}
	\scriptsize
	Consider the externally imposed shear flow 
	\begin{figure}
		\centering
		\begin{equation}
			\mathbf{u}=\begin{pmatrix}
				u(y) \\
				0 \\
				0
			\end{pmatrix}, \quad \nabla_{\mathbf{x}} \mathbf{u}=\begin{pmatrix}
				0 & 1 & 0 \\
				0 & 0 & 0 \\
				0 & 0 & 0
			\end{pmatrix}
		\end{equation}
	\end{figure}

\begin{figure}
	\small
	\begin{minipage}{0.43\textwidth}
		\includegraphics[scale=0.35]{Bilder/shearflow_12th_dr0.1}
	\end{minipage}
	\hfill 
	\begin{minipage}{0.43\textwidth}
		\includegraphics[scale=0.35]{Bilder/shearflow_12th_linspace(0,10,500)_dr=1}
	\end{minipage}
	\caption{Steady state solution of the Smoluchowski equation approximated at $T = 10$ using different values of $D_r$.}
\end{figure}
\end{frame}

\subsubsection{Stability analysis}
\begin{frame}{Shear flow: Stability analysis}
	\begin{figure}
		\centering
		\begin{minipage}{0.4\linewidth}
			\includegraphics[scale=.42]{Bilder/Stability_analysis_shearflow_2nd_dr0.1}
		\end{minipage}
		\hspace{1cm}
		\begin{minipage}{0.4\linewidth}
			\centering
			\includegraphics[scale=0.42]{Bilder/Stability_analysis_shearflow_2nd_dr1}
		\end{minipage}
		\caption{Eigenvalue of matrix $A$ with basis functions of $2nd.$ order}
	\end{figure}
	
	\begin{block}{Proposition 1}
		Sprectal method for $2nd.$ order is stable with for both small and large $D_r$.
	\end{block}
\end{frame}

\begin{frame}{Shear flow: Stability analysis}
	\begin{figure}
		\centering
		\begin{minipage}{0.4\linewidth}
			\includegraphics[scale=.42]{Bilder/Stability_analysis_shearflow_8th_dr0.1}
		\end{minipage}
		\hspace{1cm}
		\begin{minipage}{0.4\linewidth}
			\centering
			\includegraphics[scale=0.42]{Bilder/Stability_analysis_shearflow_8th_dr1}
		\end{minipage}
		\caption{Eigenvalue of matrix $A$ with basis functions of $8th.$ order}
	\end{figure}
	
	\begin{block}{Proposition 2}
		Sprectal method for $8th.$ order is stable with for both small and large $D_r$.
	\end{block}
\end{frame}

%%%%%%%%%%%%%%%%%%%%%%%%%%%%%%%%
%%%%%%%%%%%%%%%%%%%%%%%%%%%%%%%%
% Elongational flow
%%%%%%%%%%%%%%%%%%%%%%%%%%%%%%%%
%%%%%%%%%%%%%%%%%%%%%%%%%%%%%%%%

\begin{frame}
	\centering
	Elongational flow
\end{frame}


\subsection{Elongational flow}
\begin{frame}{Elongational flow}
	\scriptsize
Consider the externally imposed velocity gradient
\begin{align*}
	\nabla_{\vec{x}} \vec{u}_{\mathrm{ext}}=\left(\begin{array}{ccc}
		2 & 0 & 0 \\
		0 & -1 & 0 \\
		0 & 0 & -1
	\end{array}\right). %, \text{for $\ell = 1, -1$}.
\end{align*}

The exact steady-state solution has the form
\begin{align}
	f_{\text {exact}}(\phi, \theta)=C_1 \exp \left(-\frac{3}{2 D_{\mathrm{r}}}\left(1-\cos ^2(\phi) \sin ^2(\theta)\right)\right),
\end{align}
with the constants $C_1 = 2.30121384511755303190$ for $D_r=0.1$.

\begin{figure}
	%	\small
	%Solution on $S^2$
	\begin{minipage}{0.4\textwidth}
		\includegraphics[scale=0.35]{Bilder/exakteLsg_example3.1}
	\end{minipage}
	\hfill 
	\begin{minipage}{0.4\textwidth}
		\includegraphics[scale=0.35]{Bilder/exakteLsg_example31_l=1_dr1}
	\end{minipage}
	\caption{Exact steady state solution with different $D_r$}
\end{figure}

\end{frame}



\begin{frame}{Elongational flow}
\begin{figure}
	\begin{minipage}{0.48\textwidth}
		\includegraphics[scale=0.4]{Bilder/example3.1_14nd_linspace(0,10,500)_dr=0.1}
	\end{minipage}
	\hfill 
	\begin{minipage}{0.48\textwidth}
		\includegraphics[scale=0.4]{Bilder/example3.1_14nd_linspace(0,10,1000)_dr=1}
	\end{minipage}
	\caption{Numerical solution on $S^2$ with basis functions of $14th.$ order}
\end{figure}
\end{frame}

\begin{frame}{Convergence Study}
	For the convergence study, the maximum norm error is used
	\begin{align*}
		E_{max} = max|U_{exact} - U_{approx}|.
	\end{align*}

	\begin{figure}
		\begin{minipage}{0.45\textwidth}
			\includegraphics[scale=0.42]{Bilder/Konvergenzstudie_maxnorm_dr=0.1_l=1}
		\end{minipage}
		\hfill 
		\begin{minipage}{0.45\textwidth}
			\includegraphics[scale=0.42]{Bilder/Konvergenzstudie_maxnorm_dr=1_l=1}
		\end{minipage}
		\caption{Error with respect to basisfunction with different $D_r$}
	\end{figure}
\end{frame}

\subsubsection{Stability analysis}

\begin{frame}{Stability analysis}
	\begin{figure}
		\centering
		\subfloat[Coefficients as function order]{\includegraphics[width=5.5cm,height=4cm]{Bilder/elongational_l=1_lsg_ueber_Zeit_2nd}}
		\qquad
		\subfloat[Eigenwert of matrix $A$ order]{\includegraphics[width=5.5cm,height=3.8cm]{Bilder/Stability_analysis_2nd_dr0.1}}
		\caption{With $2nd.$ order and $D_r$ = 0.1}
	\end{figure}
\end{frame}

\begin{frame}{Stability analysis}
	\begin{figure}
		\centering
		\subfloat[Coefficients as function order]{\includegraphics[width=5.5cm,height=4cm]{Bilder/elongational_l=1_lsg_ueber_Zeit_2nd_dr=1}}
		\qquad
		\subfloat[Eigenwert of matrix $A$ order]{\includegraphics[width=5.5cm,height=3.8cm]{Bilder/Stability_analysis_2nd_dr1}}
		\caption{With $2nd.$ order and $D_r$ = 1}
	\end{figure}
\end{frame}

\begin{frame}{Stability analysis}
	\begin{figure}
		\centering
		\subfloat[Coefficients as function order]{\includegraphics[width=5.5cm,height=4cm]{Bilder/elongational_l=1_lsg_ueber_Zeit_4th}}
		\qquad
		\subfloat[Eigenwert of matrix $A$ order]{\includegraphics[width=5.5cm,height=3.8cm]{Bilder/Stability_analysis_4th_dr0.1}}
		\caption{With $4th.$ order and $D_r$ = 0.1}
	\end{figure}
\end{frame}

\begin{frame}{Stability analysis}
	\begin{figure}
		\centering
		\subfloat[Coefficients as function order]{\includegraphics[width=5.5cm,height=4cm]{Bilder/elongational_l=1_lsg_ueber_Zeit_4th_dr=1}}
		\qquad
		\subfloat[Eigenwert of matrix $A$ order]{\includegraphics[width=5.5cm,height=3.8cm]{Bilder/Stability_analysis_4th_dr1}}
		\caption{With $4th.$ order and $D_r$ = 1}
	\end{figure}
\end{frame}

\begin{frame}{Stability analysis}
	\begin{figure}
		\centering
		\subfloat[Coefficients as function order]{\includegraphics[width=5.5cm,height=4cm]{Bilder/elongational_l=1_lsg_ueber_Zeit_14th}}
		\qquad
		\subfloat[Eigenwert of matrix $A$ order]{\includegraphics[width=5.5cm,height=3.8cm]{Bilder/Stability_analysis_14th_dr0.1}}
		\caption{With $14th.$ order and $D_r$ = 0.1}
	\end{figure}
\end{frame}

\begin{comment}
\begin{frame}{Stability analysis}
	\begin{figure}
		\begin{subfigure}{0.48\textwidth}
			\includegraphics[width=\linewidth]{Bilder/elongational_l=1_lsg_ueber_Zeit_2nd}
		\end{subfigure}
		\hfill
		\begin{subfigure}{0.48\textwidth}
			\includegraphics[width=\linewidth]{Bilder/elongational_l=1_lsg_ueber_Zeit_2nd_dr=1}
		\end{subfigure}
		\caption{Coefficients as function of $2nd.$ order with different $D_r$}
	\end{figure}
\end{frame}

\begin{frame}{Stability analysis}
	\begin{figure}
		\begin{subfigure}{0.48\textwidth}
			\includegraphics[width=\linewidth]{Bilder/Stability_analysis_2nd_dr0.1}
		\end{subfigure}
		\hfill
		\begin{subfigure}{0.48\textwidth}
			\includegraphics[width=\linewidth]{Bilder/Stability_analysis_2nd_dr1}
		\end{subfigure}
		\caption{Eigenwert of matrix $A$ of $2nd.$ order with different $D_r$}
	\end{figure}
\end{frame}


\begin{frame}
	\begin{figure}
		\begin{subfigure}{0.48\textwidth}
			\includegraphics[width=\linewidth]{Bilder/elongational_l=1_lsg_ueber_Zeit_4th}
		\end{subfigure}
		\hfill
		\begin{subfigure}{0.48\textwidth}
			\includegraphics[width=\linewidth]{Bilder/elongational_l=1_lsg_ueber_Zeit_4th_dr=1}
		\end{subfigure}
		\caption{Coefficients as function of $4th.$ order}
	\end{figure}

	\begin{figure}
		\begin{subfigure}{0.43\textwidth}
			\includegraphics[width=\linewidth]{Bilder/Stability_analysis_4th_dr0.1}
		\end{subfigure}
		\hfill
		\begin{subfigure}{0.43\textwidth}
			\includegraphics[width=\linewidth]{Bilder/Stability_analysis_4th_dr1}
		\end{subfigure}
		\caption{Eigenwert of matrix $A$}
	\end{figure}
\end{frame}
\end{comment}

\begin{comment}
\begin{frame}
	\begin{figure}
		\begin{subfigure}{0.48\textwidth}
			\includegraphics[width=\linewidth]{Bilder/elongational_l=1_lsg_ueber_Zeit_6th}
		\end{subfigure}
		\hfill
		\begin{subfigure}{0.48\textwidth}
			\includegraphics[width=\linewidth]{Bilder/elongational_l=1_lsg_ueber_Zeit_6th_dr=1}
		\end{subfigure}
		\caption{Coefficients as function of $6th.$ order}
	\end{figure}

	\begin{figure}
		\begin{subfigure}{0.43\textwidth}
			\includegraphics[width=\linewidth]{Bilder/Stability_analysis_6th_dr0.1}
		\end{subfigure}
		\hfill
		\begin{subfigure}{0.43\textwidth}
			\includegraphics[width=\linewidth]{Bilder/Stability_analysis_6th_dr1}
		\end{subfigure}
		\caption{Eigenwert of matrix $A$}
	\end{figure}
\end{frame}


\begin{frame}{Elongational flow: Stability analysis}
	\begin{figure}
		\begin{subfigure}{0.48\textwidth}
			\includegraphics[width=\linewidth]{Bilder/elongational_l=1_lsg_ueber_Zeit_14th}
		\end{subfigure}
		\hfill
		\begin{subfigure}{0.48\textwidth}
			\includegraphics[width=\linewidth]{Bilder/elongational_l=1_lsg_ueber_Zeit_14th_dr=1}
		\end{subfigure}
		\caption{Coefficients as function of $14th.$ order with different $D_r$}
	\end{figure}
\end{frame}
\end{comment}

\begin{comment}
\begin{frame}{Elongational flow: Stability analysis}
	\begin{columns}
		\begin{column}{0.5\textwidth}
			\begin{figure}
				\includegraphics[width=0.97\linewidth]{Bilder/elongational_l=1_lsg_ueber_Zeit_2nd}
				\caption{Coefficients as function of time}
			\end{figure}
		\end{column}
		\begin{column}{0.5\textwidth}
			\begin{figure}
				\includegraphics[width=0.8\linewidth]{Bilder/Stability_analysis_2nd_dr0.1}
				\caption{Eigenvalue of matrix $A$}
			\end{figure}
		\end{column}
	\end{columns}
	\centering
	With $2nd.$ order and $D_r = 0.1$.
\end{frame}

\begin{frame}{Elongational flow: Stability analysis}
	\begin{columns}
		\begin{column}{0.5\textwidth}
			\begin{figure}
				\includegraphics[width=0.97\linewidth]{Bilder/elongational_l=1_lsg_ueber_Zeit_4th}
				\caption{Coefficients as function of time}
			\end{figure}
		\end{column}
		\begin{column}{0.5\textwidth}
			\begin{figure}
				\includegraphics[width=0.8\linewidth]{Bilder/Stability_analysis_4th_dr0.1}
				\caption{Eigenvalue of matrix $A$}
			\end{figure}
		\end{column}
	\end{columns}
	\centering
	With $4th.$ order and $D_r = 0.1$.
\end{frame}


\begin{frame}{Elongational flow: Stability analysis}
	\begin{columns}
		\begin{column}{0.5\textwidth}
			\begin{figure}
				\includegraphics[width=0.97\linewidth]{Bilder/elongational_l=1_lsg_ueber_Zeit_6th}
				\caption{Coefficients as function of time}
			\end{figure}
		\end{column}
		\begin{column}{0.5\textwidth}
			\begin{figure}
				\includegraphics[width=0.82\linewidth]{Bilder/Stability_analysis_6th_dr0.1}
				\caption{Eigenvalue of matrix $A$}
			\end{figure}
		\end{column}
	\end{columns}
	\centering
	With $6th.$ order and $D_r = 0.1$.
\end{frame}

\begin{frame}{Elongational flow: Stability analysis}
	\begin{columns}
		\begin{column}{0.5\textwidth}
			\begin{figure}
				\includegraphics[width=0.97\linewidth]{Bilder/elongational_l=1_lsg_ueber_Zeit_14th}
				\caption{Coefficients as function of time}
			\end{figure}
		\end{column}
		\begin{column}{0.5\textwidth}
			\begin{figure}
				\includegraphics[width=0.83\linewidth]{Bilder/Stability_analysis_14th_dr0.1}
				\caption{Eigenvalue of matrix $A$}
			\end{figure}
		\end{column}
	\end{columns}
	\centering
	With $14th.$ order and $D_r = 0.1$.
\end{frame}
\end{comment}

\begin{comment}
\begin{frame}{Elongational flow: Stability analysis}
	\begin{figure}
		\centering
		\begin{minipage}{0.4\linewidth}
			\includegraphics[scale=.42]{Bilder/Stability_analysis_2nd_dr1}
		\end{minipage}
		\hspace{1cm}
		\begin{minipage}{0.4\linewidth}
			\centering
			\includegraphics[scale=0.42]{Bilder/Stability_analysis_2nd_dr0.14}
		\end{minipage}
		\caption{Eigenvalue of matrix $A$ with basis functions of $2nd.$ order}
	\end{figure}
	
	\begin{block}{Finding 1}
		The maximum value of eigenvalue of matrix $A$ with basis functions of $2nd.$ order is $0.01714 > 0$. It follows:
		Sprectal method for $2nd.$ order is unstable with $D_r < 0.15$.
	\end{block}
\end{frame}
\end{comment}


\begin{frame}{Conclusion}
	\begin{block}{Finding}
		\begin{itemize}
			\item For small $D_r$ we have seen that the error was relatively large.
			\item The eigenvalue of the matrix $A$ of the $2nd.$ order ODE system for $D_r = 0.1$ has a positive real part $\rightarrow$
			the ansatzfunction with $2nd.$ order is unsuitable.
			\item 	The spectral method becomes stable when taking more approach functions for small $D_r$.
		\end{itemize}
	\end{block}
\end{frame}


%\begin{frame}
%	\begin{figure}[h]
%		\centering
%		\includegraphics[scale=.35]{Bilder/Solution_over_time_14th}
%		\caption{Numerical solution of ODE systems over time with $D_r$ = 1}
%		\label{fig: Solution of ODE systems over time with $D_r$ = 1}
%	\end{figure}
%\end{frame}


%\section{Example}
\begin{frame}
	\scriptsize	With velocity gradient
	\begin{align*}
		\scriptsize \nabla_{\vec{x}} \vec{u}_{ext} = \left( \begin{array}{rrr}
			0 & 1 & 0 \\ 
			0 & 0 & 0 \\
			0 & 0 & 0 \\ 
		\end{array}\right)
	\end{align*}
	\begin{figure}
	\subfloat[]{\includegraphics[scale=0.35]{Bilder/shearflow_4th_linspace(0,10,500)}}
	\qquad
	\subfloat[]{\includegraphics[scale=0.35]{Bilder/shearflow_4th_linspace(0,10,500)_dr=1}}
	\scriptsize \caption{Numerical solution of the Smoluchowski equation with Spectral method with harmonic polynomial basis functions $a)$ $b)$ $4th$.}
	\end{figure}
\end{frame}


\begin{frame}
	\begin{minipage}{0.4\textwidth}
		\includegraphics[scale=0.4]{Bilder/shearflow_10th_linspace(0,10,500)}
	\end{minipage}
	\hfill 
	\begin{minipage}{0.4\textwidth}
		\includegraphics[scale=0.4]{Bilder/shearflow_10th_linspace(0,10,500)_dr=1}
	\end{minipage}
\end{frame}







\begin{frame}
	\centering
	Thank you for listening!
\end{frame}

\AtBeginSection[]
%{
% \begin{frame}<beamer>
% \frametitle{Table of Contents}
% \tableofcontents[currentsection]
% \end{frame}
%}



%Ende
%\begin{frame}{Ende}
%	\centering
%	Thank you for listening!
%\end{frame}


\begin{frame}[allowframebreaks]
  \frametitle{Literatur}
  \nocite{*} %es werden auch Quellen aufgefuehrt, die nicht in den Folien zitiert sind.
  \bibliography{Bibliothek}
  \bibliographystyle{abbrv}
\end{frame}

\end{document}