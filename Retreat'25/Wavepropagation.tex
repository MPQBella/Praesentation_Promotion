\section{Wave Propagation Algorithm}

\begin{frame}{Approximation of Coupled Shear Flow Problem}
	\scriptsize
	Consider
	\begin{equation}
		\begin{split}
			&\partial_t Q(x,t) + A\partial_x Q(x,t) =  D(w_x)Q(x,t)+ D_rEQ(x,t) \\
			&\partial_{t}w = \partial_{xx}w + \delta(\bar{\rho}-2\sqrt{\pi} c^0_0(x,t)).
		\end{split}
		\label{coupledsys_1d}
	\end{equation}
	\begin{table}[h]
		\centering
		\renewcommand{\arraystretch}{1.3}
		\scalebox{0.95}{
			\begin{tabular}{|c|l|}
				\hline
				1. & $\frac{1}{2} \Delta t$ step on $\partial_t Q(x, t) = (D(w_x(x,t_n))+ D_rE)Q(x,t)$ \\
				\hline
				2. & $\frac{1}{4} \Delta t$ step on $\partial_t w(x, t) = \delta(\bar{\rho} - 2\sqrt{\pi} c^0_0(x,t))$ \\
				3. & $\frac{1}{2} \Delta t$ step on $\partial_t w(x, t) = \partial_{xx} w(x, t)$ \\
				4. & $\frac{1}{4} \Delta t$ step on $\partial_t w(x, t) = \delta(\bar{\rho} -2\sqrt{\pi} c^0_0(x,t))$ \\
				\hline
				5. & $\Delta t$ step on $\partial_t Q(x, t) + A \partial_x Q(x, t) = 0$ \\
				\hline
				6. & $\frac{1}{4} \Delta t$ step on $\partial_t w(x, t) = \delta(\bar{\rho} - 2\sqrt{\pi} c^0_0(x,t))$ \\
				7. & $\frac{1}{2} \Delta t$ step on $\partial_t w(x, t) = \partial_{xx} w(x, t)$ \\
				8. & $\frac{1}{4} \Delta t$ step on $\partial_t w(x, t) = \delta(\bar{\rho} -2\sqrt{\pi} c^0_0(x,t))$ \\
				\hline
				9. & $\frac{1}{2} \Delta t$ step on $\partial_t Q(x, t) = (D(w_x(x,t_{n+1}))+ D_rE)Q(x,t)$ \\
				\hline
			\end{tabular}
		}
		\caption{Splitting algorithm for solving the coupled shear flow problem (Dahm et al.)}
	\end{table}
	We use an ODE solver for $1.+9.$, LeVeque’s high resolution wave propagation algorithm for $5.$ and finite difference methods for the evolution of $w$.
\end{frame}

%-------------------------------------------------------
%-------------------------------------------------------

\begin{frame}{1D Wave Propagation Algorithm (LeVeque et al.)}
	\scriptsize
	
	We consider the hyperbolic system
	\[
	q_t + f(q)_x = 0, \quad q \in \mathbb{R}^m
	\]
	
	\begin{itemize}
		\item Discretize the domain into cells $[x_{i-1/2}, x_{i+1/2}]$.
		\item At each interface $x_{i+1/2}$, solve the Riemann problem:
		\[
		q(x,0) = 
		\begin{cases} 
			q_i, & x < x_{i+1/2}, \\ 
			q_{i+1}, & x > x_{i+1/2}.
		\end{cases}
		\]
		\item Decompose the jump into waves along eigenvectors of $A = f'(q)$:
		\[
		q_{i+1} - q_i = \sum_{p=1}^m \alpha^p r^p, 
		\quad W^p = \alpha^p r^p, \; \text{travelling with } s^p.
		\]
		\item Define fluctuations:
		\[
		\mathcal{A}^+ \Delta q_{i-1/2} = \sum_{p: s^p>0} s^p W^p, \quad
		\mathcal{A}^- \Delta q_{i+1/2} = \sum_{p: s^p<0} s^p W^p
		\]
		\item Update cell averages:
		\[
		Q_i^{n+1} = Q_i^n - \frac{\Delta t}{\Delta x} 
		\Big( \mathcal{A}^+ \Delta q_{i-1/2} + \mathcal{A}^- \Delta q_{i+1/2} \Big)
		\]
	\end{itemize}
\end{frame}



%-------------------------------------------------------
%-------------------------------------------------------

\begin{frame}{Multidimensional Wave Propagation (2D and 3D)}
	\scriptsize
	\begin{columns}[c]
		
		%------------------ Linke Spalte: Theorie ------------------%
		\begin{column}{0.50\textwidth}
			\textbf{General system:}
			\[
			q_t + f(q)_x + g(q)_y + h(q)_z = 0
			\]
			
			\textbf{Key ideas:}
			\begin{itemize}
				\item \textbf{1D Riemann problems at cell interfaces:}  
				Solve along the normal direction of each cell interface.
				\item \textbf{Wave decomposition:}  
				Decompose jumps into waves \(W^p\) with speeds \(s^p\).
				\item \textbf{Fluctuations:}  
				\[
				\mathcal{A}^\pm \Delta q, \quad 
				\mathcal{B}^\pm \Delta q, \quad 
				\mathcal{C}^\pm \Delta q
				\]
				\item \textbf{Transverse propagation:}  
				\begin{itemize}
					\item 2D: Waves in \(x\)-direction generate waves in \(y\), and vice versa.
					\item 3D: Each wave generates transverse waves along in the other two directions.
				\end{itemize}
			\end{itemize}
		\end{column}
		
		%------------------ Rechte Spalte: Würfel ------------------%
		\begin{column}{0.50\textwidth}
			\centering
         % 1D-Gitter
       % \begin{tikzpicture}[scale=0.8]
 	   %  \foreach \i in {0,1,2} {\draw (\i,0) rectangle (\i+1,0.8);}
 	   %  \node at (1.5, 1) {1D grid};
 	    % \node at (0.5,0.4) {$i-1$};
 	    % \node at (1.5,0.4) {$i$};
 	   %  \node at (2.5,0.4) {$i+1$};
 	    % \node[below] at (1,0) {$x_{i-1/2}$};
 	   %  \node[below] at (2,0) {$x_{i+1/2}$};
       % \end{tikzpicture}
 
      %  \vspace{0.5cm} % Abstand zwischen den Grafiken
 
        % 2D-Gitter
       \begin{tikzpicture}[scale=0.8]
       	% Titel
       	\node at (2,4) {2D grid};
       % nur äußeres Quadrat
        \draw (0,0) rectangle (4,4);
      
       % Beschriftungen
        \node at (1,1) {$(i,j)$};
        \node at (3,1) {$(i+1,j)$};
        \node at (1,3) {$(i,j+1)$};
        \node at (3,3) {$(i+1,j+1)$};
      
       % Schnittlinien
        \draw[dashed] (2,0) -- (2,4) node[midway,right]
        \draw[dashed] (0,2) -- (4,2)node[midway,above]
       
       % optionale äußere Beschriftungen
        \node[left] at (0,0.5) {$y_{j-1/2}$};
        \node[below] at (0.5,0) {$x_{i-1/2}$};
 	
      \end{tikzpicture}
	 \end{column}
	\end{columns}
\end{frame}


\begin{comment}
\begin{frame}{Multidimensional Wave Propagation (2D and 3D)}
	\scriptsize
	General system:
	\[
	q_t + f(q)_x + g(q)_y + h(q)_z = 0.
	\]
	
	\begin{itemize}
		\item At each cell face, solve a 1D Riemann problem in the normal direction.
		\item Decompose the jump into waves $W^p$ and corresponding speeds $s^p$.
		\item Compute fluctuations:
		\[
		\mathcal{A}^\pm \Delta q, \quad 
		\mathcal{B}^\pm \Delta q, \quad 
		\mathcal{C}^\pm \Delta q.
		\]
		\item Waves also propagate transversely:
		\begin{itemize}
			\item In 2D: a wave in $x-$direction can propagates in $y$-direction, and vice versa.
			\item In 3D: each wave generates transverse waves in the other two directions.
		\end{itemize}
	\end{itemize}
\end{frame}
\end{comment}

%-------------------------------------------------------
%-------------------------------------------------------

\begin{frame}{3D Transverse and Double Transverse Propagation}
	\scriptsize
	
    In 3D: every wave generates transverse and double–transverse contributions in the other directions

\vspace{0.5em} 
Examples of Transverse Interactions
		\begin{itemize}
			\item $x \to y$ (transverse) 
			\item $x \to z$ (transverse)
			\item $x \to y \to z$ (double transverse)
			\item $y \to x \to z$, etc.
		\end{itemize}

	\begin{block}{Update Scheme}
	\[
	\begin{aligned}
		Q_{i,j,k}^{n+1} &= Q_{i,j,k}^n
		- \frac{\Delta t}{\Delta x} \Big( \mathcal{A}^+ \Delta q_{i-1/2,j,k}
		+ \mathcal{A}^- \Delta q_{i+1/2,j,k} \Big) \\
		&\quad - \frac{\Delta t}{\Delta y} \Big( \mathcal{B}^+ \Delta q_{i,j-1/2,k}
		+ \mathcal{B}^- \Delta q_{i,j+1/2,k} \Big) \\
		&\quad - \frac{\Delta t}{\Delta z} \Big( \mathcal{C}^+ \Delta q_{i,j,k-1/2}
		+ \mathcal{C}^- \Delta q_{i,j,k+1/2} \Big).
	\end{aligned}
	\]
\end{block}
\end{frame}

%-------------------------------------------------------
%-------------------------------------------------------

\begin{comment}
\begin{frame}{Wave Propagation}
	\scriptsize
	The scheme can be specified by three integers $(m_1, m_2, m_3)$, which represent the following [LeVeque et. al]:
	\begin{align*}
		m_1 &= \left\{
		\begin{array}{ll}
			1, & \parbox[t]{0.6\textwidth}{The second order correction wave is not included,\\
				thus the method is formally first order accurate.} \\
			2, & \text{The correction wave is included.}
		\end{array}
		\right. \\
		m_2 &= \left\{
		\begin{array}{lll}
			0, & \text{No transverse propagation.} \\
			1, & \text{The propagation of the increment wave.} \\
			2, & \text{Transverse propagation of both increment and correction wave.}
		\end{array}
		\right. \\
		m_3 &= \left\{
		\begin{array}{lll}
			0, & \text{No double transverse propagation.} \\
			1, & \text{Double transverse propagation of the increment wave.} \\
			2, &  \parbox[t]{0.6\textwidth}{Double transverse propagation of both increment and \\
				correction wave.}
		\end{array}
		\right.
	\end{align*}
\end{frame}
\end{comment}