\section{Approximations of the Coupled Model}
\begin{frame}{Approximation of Coupled Shear Flow Problem}
	\scriptsize
	Consider
	\begin{equation}
		\begin{split}
			&\partial_t Q(x,t) + A\partial_x Q(x,t) =  D(w_x)Q(x,t)+ D_rEQ(x,t) \\
			&\partial_{t}w = \partial_{xx}w + \delta(\bar{\rho}-2\sqrt{\pi} c^0_0(x,t)).
		\end{split}
		\label{coupledsys_1d}
	\end{equation}
	\begin{table}[h]
		\centering
		\renewcommand{\arraystretch}{1.3}
		\scalebox{0.95}{
			\begin{tabular}{|c|l|}
				\hline
				1. & $\frac{1}{2} \Delta t$ step on $\partial_t Q(x, t) = (D(w_x(x,t_n))+ D_rE)Q(x,t)$ \\
				\hline
				2. & $\frac{1}{4} \Delta t$ step on $\partial_t w(x, t) = \delta(\bar{\rho} - 2\sqrt{\pi} c^0_0(x,t))$ \\
				3. & $\frac{1}{2} \Delta t$ step on $\partial_t w(x, t) = \partial_{xx} w(x, t)$ \\
				4. & $\frac{1}{4} \Delta t$ step on $\partial_t w(x, t) = \delta(\bar{\rho} -2\sqrt{\pi} c^0_0(x,t))$ \\
				\hline
				5. & $\Delta t$ step on $\partial_t Q(x, t) + A \partial_x Q(x, t) = 0$ \\
				\hline
				6. & $\frac{1}{4} \Delta t$ step on $\partial_t w(x, t) = \delta(\bar{\rho} - 2\sqrt{\pi} c^0_0(x,t))$ \\
				7. & $\frac{1}{2} \Delta t$ step on $\partial_t w(x, t) = \partial_{xx} w(x, t)$ \\
				8. & $\frac{1}{4} \Delta t$ step on $\partial_t w(x, t) = \delta(\bar{\rho} -2\sqrt{\pi} c^0_0(x,t))$ \\
				\hline
				9. & $\frac{1}{2} \Delta t$ step on $\partial_t Q(x, t) = (D(w_x(x,t_{n+1}))+ D_rE)Q(x,t)$ \\
				\hline
			\end{tabular}
		}
		\caption{Splitting algorithm for solving the coupled shear flow problem (Dahm et al.)}
	\end{table}
	We use an ODE solver for $1.+9.$, LeVeque’s high resolution wave propagation algorithm for $5.$ and finite difference methods for the evolution of $w$.
\end{frame}


%----------------------------------------------------------


\begin{frame}{Numerical Result for Shear Flow}
		\scriptsize
		Let 
		\begin{align*}
			c^0_0 (x,0) = (1+(1\cdot10^{-4} \cdot \eta(x)-5\cdot10^{-5}))/(2\sqrt{\pi}),
		\end{align*}
		where $\eta(x)$ is a random variable taking values in the interval $\pm \frac{1}{2}$.
		
		\begin{figure}
			\centering
			\begin{minipage}{0.46\textwidth}
				\includegraphics[width=\textwidth]{Bilder_wx/ClusterFormation/N=1vsN=6_Dr=0.05_mx=8192}
			\end{minipage}
			\hfill
			\begin{minipage}{0.46\textwidth}
				\includegraphics[width=\textwidth]{Bilder_wx/ClusterFormation/N=2vsN=6_Dr=0.05_mx=8192}
			\end{minipage}
			\caption{Approximation of the coupled problem for shear flow with $D_r =0.05$. The plot shows the density at time $t=30$ for $N = 1$ and $N = 2$ (blue line). A reference solution is calculated with $N = 6$ moment equations (black line).}
			\label{ClusterFormation}
		\end{figure}
		%The Figure (\ref{ClusterFormation}) shows that using $N = 1$ moment equations, compared to the reference solution with $N = 6$, already provides a good approximation of cluster formation. The numerical solution for $N=2$ aligns very well with the reference solution.
\end{frame}


%----------------------------------------------------------


\begin{frame}{Approximations of the Two-Dimensional Hyperbolic System}
		\scriptsize
		\begin{figure}[H]
			\centering
			\begin{minipage}{0.4\textwidth}
				\includegraphics[scale=0.23]{Bilder_wxwy/Sol_onSphere_wx=1=wy_Dr=1_N=7}
			\end{minipage}
			\hfill 
			\begin{minipage}{0.4\textwidth}
				\includegraphics[scale=0.23]{Bilder_wxwy/Sol_onSphere_wx=-1=wy_Dr=1_N=7}
			\end{minipage}
			\caption{Numerical solution of the drift-diffusion term for fixed $w_x$ and $w_y$.}
		\end{figure}
		\begin{figure}[H]
			\centering
			\begin{minipage}{0.4\textwidth}
				\includegraphics[scale=0.23]{Bilder_wxwy/t=0_wxwy=1_wxwy=-1}
			\end{minipage}
			\hfill 
			\begin{minipage}{0.4\textwidth}
				\includegraphics[scale=0.23]{Bilder_wxwy/t=200_wxwy=1_wxwy=-1}
			\end{minipage}
			\caption{Numerical results for $c^0_0$ at different times using $D_r=1$ and $w_x=w_y=1$ for $x<50$ and $w_x=w_y=-1$ otherwise. A cluster with higher particle density splits into two, each moving in opposite directions.}
		\end{figure}
\end{frame}
